%% ------------------------------------------------------------------------- %%
\chapter{Introdução}
\label{cap:introducao}


Esse trabalho surge de uma colaboração entre a empresa Intercement e o Laboratório de Inteligência Artificial e Métodos Formais do IME-USP. Foram concedidos 10 anos de dados de diversas etapas da produção de cimento de uma fábrica. Esse documento apresenta um estudo com a análise desses dados, desde a sua limpeza até uma criação de modelos preditivos para os mesmos.

Os dados foram primeiramente convertidos para formato \textbf{csv} e então importados para o ambiente Python, usando as bibliotecas pandas, matplotlib, numpy, keras e sklearn.


% \emph{Thesis are random access. Do NOT feel obliged to read a thesis from beginning to end.}



%% ------------------------------------------------------------------------- %%
\section{Considerações Preliminares}
\label{sec:consideracoes_preliminares}

Considerações preliminares\footnote{Nota de rodapé (não abuse).}\index{genoma!projetos}.
% index permite acrescentar um item no indice remissivo
Texto texto texto texto texto texto texto texto texto texto texto texto texto
texto texto texto texto texto texto texto texto texto texto texto texto texto
texto texto texto texto texto texto texto.
 

%% ------------------------------------------------------------------------- %%
\section{Objetivos}
\label{sec:objetivo}

Texto texto texto texto texto texto texto texto texto texto texto texto texto
texto texto texto texto texto texto texto texto texto texto texto texto texto
texto texto texto texto texto texto.

%% ------------------------------------------------------------------------- %%
\section{Contribuições}
\label{sec:contribucoes}

As principais contribuições deste trabalho são as seguintes:

\begin{itemize}
  \item Item 1. Texto texto texto texto texto texto texto texto texto texto
  texto texto texto texto texto texto texto texto texto texto.

  \item Item 2. Texto texto texto texto texto texto texto texto texto texto
  texto texto texto texto texto texto texto texto texto texto.

\end{itemize}

%% ------------------------------------------------------------------------- %%
\section{Organização do Trabalho}
\label{sec:organizacao_trabalho}

No Capítulo~\ref{cap:conceitos}, apresentamos os conceitos ... Finalmente, no
Capítulo~\ref{cap:conclusoes} discutimos algumas conclusões obtidas neste
trabalho. Analisamos as vantagens e desvantagens do método proposto ... 

As sequências testadas no trabalho estão disponíveis no Apêndice \ref{ape:sequencias}.
