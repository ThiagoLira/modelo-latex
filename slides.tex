%%%%%%%%%%%%%%%%%%%%%%%%%%%%%%%%%%%%%%%%%
% Beamer Presentation
% LaTeX Template
% Version 1.0 (10/11/12)
%
% This template has been downloaded from:
% http://www.LaTeXTemplates.com
%
% License:
% CC BY-NC-SA 3.0 (http://creativecommons.org/licenses/by-nc-sa/3.0/)
%
%%%%%%%%%%%%%%%%%%%%%%%%%%%%%%%%%%%%%%%%%

%----------------------------------------------------------------------------------------
%	PACKAGES AND THEMES
%----------------------------------------------------------------------------------------

\documentclass{beamer}

\mode<presentation> {

% The Beamer class comes with a number of default slide themes
% which change the colors and layouts of slides. Below this is a list
% of all the themes, uncomment each in turn to see what they look like.

%\usetheme{default}
%\usetheme{AnnArbor}
%\usetheme{Antibes}
%\usetheme{Bergen}
%\usetheme{Berkeley}
%\usetheme{Berlin}
%\usetheme{Boadilla}
%\usetheme{CambridgeUS}
%\usetheme{Copenhagen}
%\usetheme{Darmstadt}
%\usetheme{Dresden}
%\usetheme{Frankfurt}
%\usetheme{Goettingen}
%\usetheme{Hannover}
%\usetheme{Ilmenau}
%\usetheme{JuanLesPins}
%\usetheme{Luebeck}
\usetheme{Madrid}
%\usetheme{Malmoe}
%\usetheme{Marburg}
%\usetheme{Montpellier}
%\usetheme{PaloAlto}
%\usetheme{Pittsburgh}
%\usetheme{Rochester}
%\usetheme{Singapore}
%\usetheme{Szeged}
%\usetheme{Warsaw}

% As well as themes, the Beamer class has a number of color themes
% for any slide theme. Uncomment each of these in turn to see how it
% changes the colors of your current slide theme.

%\usecolortheme{albatross}
%\usecolortheme{beaver}
%\usecolortheme{beetle}
%\usecolortheme{crane}
%\usecolortheme{dolphin}
%\usecolortheme{dove}
%\usecolortheme{fly}
%\usecolortheme{lily}
%\usecolortheme{orchid}
%\usecolortheme{rose}
%\usecolortheme{seagull}
%\usecolortheme{seahorse}
%\usecolortheme{whale}
%\usecolortheme{wolverine}

%\setbeamertemplate{footline} % To remove the footer line in all slides uncomment this line
%\setbeamertemplate{footline}[page number] % To replace the footer line in all slides with a simple slide count uncomment this line

%\setbeamertemplate{navigation symbols}{} % To remove the navigation symbols from the bottom of all slides uncomment this line
}
\newcommand{\R}{\mathbb{R}}
\usepackage{graphicx} % Allows including images
\usepackage{booktabs} % Allows the use of \toprule, \midrule and \bottomrule in tables
\usepackage[brazilian]{babel}
\usepackage[utf8]{inputenc}
\usepackage[T1]{fontenc}
\usepackage{listings}
\usepackage{hyperref}
\hypersetup{
    colorlinks=true,
    linkcolor=blue,
    filecolor=magenta,      
    urlcolor=cyan,
}

\usepackage{empheq}
\usepackage[many]{tcolorbox}
\usepackage{relsize}
\usetikzlibrary{positioning,backgrounds, fit, arrows.meta,shapes.arrows,shapes}
\usepackage{lmodern}
\usepackage{bm}
\usepackage[scale=2]{ccicons}
\usepackage{pgfplots}
\usepackage{array,colortbl,xcolor}
\usepgfplotslibrary{dateplot}
\usepackage{setspace}
\usepackage{etoolbox}
\usepackage{xspace}
\usepackage{tkz-euclide}
\usepackage{tikz}
\AtBeginEnvironment{quote}{\singlespacing}

% definitions
\input{definitions/colors}
\input{definitions/styles}

% new commands
\input{tiks/all_new_commands.tex}


\usetheme{metropolis}
\urlstyle{same}
\graphicspath{ {figuras/} }
%----------------------------------------------------------------------------------------
%	TITLE PAGE
%----------------------------------------------------------------------------------------

\title{Modelos Sequencias para Regressão em Séries Temporais} % The short title appears at the bottom of every slide, the full title is only on the title page

\author{Thiago Lira} % Your name
\institute[IME-USP] % Your institution as it will appear on the bottom of every slide, may be shorthand to save space
{
Instituto de Matemática e Estatística - USP \\ % Your institution for the title page
\medskip
\textit{thlira15@gmail.com} % Your email address
}
\date{\today} % Date, can be changed to a custom date

\begin{document}

\begin{frame}
\titlepage % Print the title page as the first slide
\end{frame}

\begin{frame}
\frametitle{Overview} % Table of contents slide, comment this block out to remove it
\tableofcontents % Throughout your presentation, if you choose to use \section{} and \subsection{} commands, these will automatically be printed on this slide as an overview of your presentation
\end{frame}


\section{Introdução}


\begin{frame}
\frametitle{Histórico}

\begin{itemize}
\item A durabilidade e vida útil do cimento tem sido o problema mais importante enfrentado
pela indústria de construção civil nas últimas décadas.
\item Prever a resistência do Cimento ainda durante a sua produção amenizaria
  custos da ordem de bilhões de dólares para a indústria.
\item Avanços recentes no campo de Deep Learning forneceram ferramentas que
  melhoram o estado da arte para modelagem de Séries Temporais.
\end{itemize}

\end{frame}



\begin{frame}
\frametitle{Etapas da Produção de Cimento}
%%% slide cimento foto do processo com numeros DECORA FDP
\begin{figure}[H]
\centering
\includegraphics[scale=0.5]{cimento.png}
\caption{Representação das diversas etapas da produção de cimento}
\end{figure}
\end{frame}

\section{Análise dos Dados}

\begin{frame}
%%% plot com numero de entradas faltando nos dados 
\begin{figure}[H]
\centering
\includegraphics[scale=0.3]{slides_dados_pct}
\caption{Dados faltantes em cada coluna dos dados de Expedição de Cimento}
\end{figure}

\end{frame}

\begin{frame}
%%% slide apenas do RC3,RC7 e RC28
\begin{figure}[H]
\centering
\includegraphics[scale=0.4]{slides_dados}
\caption{Dados que serão usados para o problema de Regressão}
\end{figure}

\end{frame}


\begin{frame}
%%% slide com tabela de correlação
\frametitle{Tabela de Correlação}
\begin{table}[H]
\centering
\begin{tabular}{l|lll}
\cline{2-4}
\textbf{}                  & \multicolumn{1}{l|}{RC3} & \multicolumn{1}{l|}{RC7} & \multicolumn{1}{l|}{RC28} \\ \hline
\multicolumn{1}{|l|}{RC3}  & 1                        & 0.734201                 & 0.388973                  \\ \cline{1-1}
\multicolumn{1}{|l|}{RC7}  & 0.734201                 & 1                        & 0.484725                  \\ \cline{1-1}
\multicolumn{1}{|l|}{RC28} & 0.388973                 & 0.484725                 & 1                         \\ \cline{1-1}
\end{tabular}
\caption{Correlação entre Índices de Resistência dos dados de
  Expedição de Cimento}
\label{corr3728}
\end{table}

\end{frame}

\begin{frame}
%%% slide com analise FFT só do RC28
\frametitle{Análise de Periodicidade}
\begin{figure}[H]
\centering
\includegraphics[scale=0.6]{FFT_RC28.png}
\caption{Análise Espectral para preditor RC28}
\end{figure}

\end{frame}


\begin{frame}
  \frametitle{Definição do Problema}
%%% slide definindo aprendizado supervisionado/regressao
  \begin{itemize}
    \item O problema de modelar o índice de Resistência é um problema de
  \textbf{Aprendizado Supervisionado} de \textbf{Regressão}.
    \item Queremos modelar uma função que aprenda com um conjunto de dados $\{(x_1,y_1)
\dots , (x_n,y_n)\}$ e possa nos gerar anotações inéditas $y^*$ para novas
entradas $x^*$.

    \item O aprendizado busca em uma certa família de funções aquela que maximiza a
      verossimilhança de uma distribuição implícita $p'(y | x)$ aprendida pelo modelo.
    
  \end{itemize}
  

  
\end{frame}

\begin{frame}
  \frametitle{Definição do Problema}

  \begin{itemize}
\item Em um primeiro momento podemos criar triplas de treino $(RC3_i,RC7_i,RC28_i)$, e queremos
  encontrar uma função: \\ 
  \begin{empheq}[box=\tcbhighmath]{align}
  \mathnormal{f}(RC3_i,RC7_i) = \widehat{RC28}_i 
  \end{empheq}
    \end{itemize}

Mas desse jeito o modelo não estaria usando nenhuma informação temporal dos dados.
    
\end{frame}


\begin{frame}
 
  \frametitle{Séries Temporais}

  \begin{itemize}
\item Dados anotados temporalmente, normalmente em intervalos regulares. O passado de
uma série pode trazer informação para prever o seu futuro.
\item Agora nosso problema
de aprendizado supervisionado leva em conta diversas anotações passadas. A
distribuição aprendida pelo modelo seria condicionada por diversas anotações
subsequentes passadas. 
    \end{itemize}
  
  \begin{empheq}[box=\tcbhighmath]{align}
  p(y | x_{t} ,x_{t -1},x_{t -2},x_{t-3} , \dots, x_{t-T})
  \end{empheq}
  %%% slide definindo serie temporal
\end{frame}

\section{Modelos Escolhidos}

\begin{frame}
%%% slide mostrando rede neural (mesmo diagrama da quali)
  \frametitle{Redes Neurais}
  \begin{empheq}[box=\tcbhighmath]{align}
    \mathnormal{f}_i (x)=  a_i = \sigma(W_i*a_{i-1} + b_i) 
  \end{empheq}

\begin{figure}
  \centering
  \input{chapters/NN.tex}
  \label{fig:nn}
\end{figure}


\end{frame}


\begin{frame}
\frametitle{Modelos Sequenciais: RNNs}
%%% slide definindo uma LSTM
\begin{figure}[H]
  \input{tiks/RNNSimplified.tex}
  \caption{RNN genérica}
\end{figure}
\end{frame}

\begin{frame}
\frametitle{Modelos Sequenciais: RNNs}
%%% slide definindo uma LSTM 2
\begin{figure}[H]
  \input{tiks/RRNSimplifiedUnrolled.tex}
  \caption{RNN genérica através do tempo}
\end{figure}

\end{frame}


\begin{frame}
\frametitle{Modelos Sequenciais: LSTMs}
%%% slide definindo uma LSTM 2
  \resizebox{1\textwidth}{!}{
    \input{chapters/lstm.tex}
    }

\end{frame}

\begin{frame}
\frametitle{Modelos Sequenciais: Encoder-Decoder}
%%% slide definindo uma LSTM 2
\begin{figure}[H]
\centering
\input{chapters/encdec.tex}
\caption{ Diagrama de Rede Encoder-Decoder}

\end{figure}

\end{frame}

\begin{frame}
%%% slide com modelo Uber
\begin{figure}[H]
\centering
\includegraphics[scale=0.4]{uber.png}
\caption{Arquitetura do modelo Encoder-Decoder-Forecaster}
\end{figure}
\end{frame}

\begin{frame}
%%% slide inferencia bayesiana com a integral e os caralho
\end{frame}

\begin{frame}
  \frametitle{Dropout como Regularizador}
  \centering
%%% slide com dropout imagem da quali
  \resizebox{1\textwidth}{!}{
      \input{chapters/dropout.tex}
    }

    O Dropout força a rede neural a distribuir o aprendizado por todos os neurônios.
    
\end{frame}

\begin{frame}
  \frametitle{Dropout como Inferência Bayesiana}
  %%% slide com dropout imagem da quali

  É demonstrado que o Dropout pode ser interpretado como considerar uma
  distribuição de probabilidade em todos os parâmetros do modelo.

  \begin{empheq}[box=\tcbhighmath]{align*}
  y^* &= \hat{a}W_2 \\
          &= (a * diag(\hat{\epsilon}_2))W_2 \\
          &=   \sigma(\hat{x}W_1 + b) *(diag(\hat{\epsilon}_2)W_2) \\
          &=   \sigma(x^* (diag(\hat{\epsilon}_1)W_1) + b) * (diag(\hat{\epsilon}_2)W_2) \\
  \end{empheq}
  Onde cada vetor $\epsilon_k$ é uma realização de um vetor de 0s e 1s seguindo
  uma distribuição de Bernoulli com probabilidade $p_k$ em cada posição: \\
  \begin{empheq}[box=\tcbhighmath]{align*}
    \hat{\epsilon}^{i}_k \sim Bernoulli(p_k)
  \end{empheq}

  
\end{frame}

\begin{frame}
  \frametitle{Dropout como Inferência Bayesiana}
  %%% slide com dropout imagem da quali
  Definindo-se: \\
  \begin{empheq}[box=\tcbhighmath]{align*}
    \hat{W_1} &:= diag(\hat{\epsilon}_1)W_1 \\
    \hat{W_2} &:= diag(\hat{\epsilon}_2)W_2
  \end{empheq}
Então podemos escrever a saída da rede neural como: \\
  \begin{empheq}[box=\tcbhighmath]{align*}
        y^*  =   \sigma(x \hat{W}_1 + b) * \hat{W}_2 =:
        f^{\hat{W}_1,\hat{W}_2,b}(x^*) \\
\end{empheq}
\end{frame}

\begin{frame}
  \frametitle{Dropout como Inferência Bayesiana}
  %%% slide com dropout imagem da quali

  Dessa maneira, podemos estimar os dois primeiros momentos das predições de
  maneira simples: \\
  
  \begin{empheq}[box=\tcbhighmath]{align*}
   \widetilde{\mathbb{E}}[y^*] &= \frac{1}{B}\sum^B_{B=1}\hat{y}^*_{(B)} \\ 
   \widetilde{\mathit{Var}}[y^*]  &= \frac{1}{B}\sum^B_{B=1}(\hat{y}^*_{(B)} - \bar{y}^*)^2 
  \end{empheq}

\end{frame}


\begin{frame}
%%% slide do modelo da uber com zoom
\end{frame}

\begin{frame}
%%% formulas de variancia no modelo uber

  Finalmente, estimamos a incerteza do modelo usando o MC Dropout. E o ruído
  inerente usando a variância amostral de um conjunto de dados de validação:\\ 

  \begin{empheq}[box=\tcbhighmath]{align*}
   \widetilde{\mathit{Var}}[f^W(x^*)]  &=
   \frac{1}{B}\sum^B_{B=1}(\hat{y}^*_{(B)} - \hat{y}^*)^2 \\ 
   \widetilde{\sigma}^2 &= \frac{1}{V}\sum^V_{V=1}(y'_v - f^W(x'_v))^2 
  \end{empheq}

\end{frame}

\begin{frame}
%%% resultados modelos não-sequenciais
\end{frame}

\begin{frame}
%%% resultados modelo uber
\end{frame}

\begin{frame}
%%% conclusão até agora
\end{frame}

\begin{frame}
%%% direção futura imagem do cronograma
\end{frame}















\end{document}