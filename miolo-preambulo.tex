%%%%%%%%%%%%%%%%%%%%%%%%%%%%%%%%%%%%%%%%%%%%%%%%%%%%%%%%%%%%%%%%%%%%%%%%%%%%%%%%
%%%%%%%%%%%%%%%%%%%%%%% CONFIGURAÇÕES E PACOTES BÁSICOS %%%%%%%%%%%%%%%%%%%%%%%%
%%%%%%%%%%%%%%%%%%%%%%%%%%%%%%%%%%%%%%%%%%%%%%%%%%%%%%%%%%%%%%%%%%%%%%%%%%%%%%%%

% Vários comandos auxiliares para o desenvolvimento de packages e classes;
% aqui, usamos em alguns comandos de formatação.
\usepackage{etoolbox}

% Detecta o tipo de sistema que estamos usando (XeTeX, LuaTeX ou pdfTeX). Na
% verdade, ifpdf não detecta pdfTeX, mas sim se estamos gerando um PDF; como
% só XeTeX, LuaTeX e pdfTeX geram PDFs, combinando todos é possível identificar
% pdfTeX também.
\usepackage{ifxetex}
\usepackage{ifluatex}
\usepackage{ifpdf}

% "fontenc" é um parâmetro interno do LaTeX. O fontenc default é OT1, mas ele
% tem algumas limitações; a mais importante é que, com ele, palavras acentuadas
% não podem ser hifenizadas. Por conta disso, quase todos os documentos LaTeX
% utilizam o fontenc T1. A escolha do fontenc tem consequências para as fontes
% que podem ser usadas no documento; hoje em dia T1 tem mais opções de
% qualidade, então não se perde nada.
\usepackage[T1]{fontenc}

% apenas útil para LaTeX tradicional e pdfTeX; XeTeX e LuaTeX usam sempre utf8.
\usepackage[utf8]{inputenc}

% Permite criar "headed lists", ou seja, "listas" de elementos que vão
% aparecendo ao longo do documento (como, por exemplo, teoremas). Podem ser
% também citações a autores específicos, seções de um documento que está
% sendo analisado etc. Precisa ser carregado antes das definições de fontes.
\usepackage{amsthm}

% Internacionalização dos nomes das seções ("chapter" X "capítulo" etc.),
% hifenização e outras convenções tipográficas. babel deve ser um dos
% primeiros pacotes carregados. É possível passar a língua do documento
% como parâmetro aqui, mas já fizemos isso ao carregar a classe, mais acima.
\usepackage{babel}

% Comandos rápidos para mudar de língua:
% \en -> muda para o inglês
% \br -> muda para o português
% \texten{blah} -> o texto "blah" é em inglês
% \textbr{blah} -> o texto "blah" é em português
\babeltags{br = brazil, en = english}

% É possível personalizar as palavras-chave que babel utiliza, por exemplo:
%\addto\extrasbrazil{\renewcommand{\refname}{Bibliografia}}

% Para línguas baseadas no alfabeto latino, como o inglês e o português,
% o pacote babel funciona muito bem, mas com outros alfabetos ele às vezes
% falha. Por conta disso, o pacote polyglossia foi criado para substituí-lo.
% polyglossia só funciona com LuaTeX e XeTeX; como babel também funciona com
% esses sistemas, provavelmente não há razão para usar polyglossia, mas é
% possível que no futuro esse pacote se torne o padrão.
%\usepackage{polyglossia}
%\setdefaultlanguage{brazil}
%\setotherlanguage{english}

% microajustes no tamanho das letras, espaçamento etc. para melhorar
% a qualidade visual do resultado. LaTeX tradicional não dá suporte a
% nenhum tipo de microajuste; pdfLaTeX dá suporte a todos. LuaLaTeX
% e XeLaTeX dão suporte a alguns:
%
% * expansion não funciona com XeLaTeX
% * tracking não funciona com XeLaTeX; é possível obter o mesmo resultado
%   com a opção "LetterSpace" do pacote fontspec, mas a configuração é
%   totalmente manual. Por padrão, aumenta o afastamento entre caracteres
%   nas fontes "small caps"; o resultado não se presta ao uso na
%   bibliografia ou citações, então melhor desabilitar.
% * kerning e spacing só funcionam com pdfLaTex; ambas são funções
%   consideradas experimentais e nem sempre produzem resultados vantajosos.

\newcommand\microtypeopts{
  protrusion=true,
  tracking=false,
  kerning=false,
  spacing=false
}

\ifxetex
  \usepackage[expansion=false,\microtypeopts]{microtype}
\else
  \usepackage[expansion=true,\microtypeopts]{microtype}
\fi

% Alguns "truques" (sujos?) para minimizar over/underfull boxes.
\tolerance=800
\hyphenpenalty=800
\setlength{\emergencystretch}{2.5em}

% Normalmente, LaTeX faz o final da página terminar sempre no mesmo lugar
% (exceto no final dos capítulos). Esse padrão pode ser ativado explicitamente
% com o comando "\flushbottom". Mas se, por alguma razão, o volume de texto na
% página é "pequeno", essa página vai ter espaços verticais artificialmente
% grandes. Uma solução para esse problema é modificar o padrão para
% "\raggedbottom"; isso permite que as páginas terminem em lugares diferentes.
% Outra opção é corrigir manualmente cada página problemática, por exemplo
% com o comando "\enlargethispage".
%\raggedbottom

% Por padrão, LaTeX coloca uma espaço aumentado após sinais de pontuação;
% Isso não é tão bom quanto alguns TeX-eiros defendem :) .
% Esta opção desabilita isso e, consequentemente, evita problemas com
% "id est" (i.e.) e "exempli gratia" (e.g.)
\frenchspacing

% LaTeX por default segue o estilo americano e não faz a indentação da
% primeira linha do primeiro parágrafo de uma seção; este pacote ativa essa
% indentação, como é o estilo brasileiro
\usepackage{indentfirst}

% LaTeX às vezes coloca notas de rodapé logo após o final do texto da
% página ao invés de no final da página; este pacote evita isso.
\usepackage[bottom]{footmisc}

% Se uma página está vazia, não imprime número de página ou cabeçalho
\usepackage{emptypage}

% Espaçamento entre linhas configurável (\singlespacing, \onehalfspacing etc.)
\usepackage{setspace}

% A primeira linha de cada parágrafo costuma ter um pequeno recuo para
% tornar mais fácil visualizar onde cada parágrafo começa. Além disso, é
% possível colocar um espaço em branco entre um parágrafo e outro. Esta
% package coloca o espaço em branco e desabilita o recuo; como queremos
% o espaço *e* o recuo, é preciso guardar o valor padrão do recuo e
% redefini-lo depois de carregar a package.
\newlength\oldparindent
\setlength\oldparindent\parindent
\usepackage[parfill]{parskip}
\setlength\parindent\oldparindent

% Carrega nomes de cores disponíveis (podem ser usados com hyperref e listings)
\usepackage[usenames,svgnames,dvipsnames]{xcolor}

% Por padrão, o algoritmo LaTeX para textos não-justificados é (muito) ruim;
% este pacote implementa um algoritmo bem melhor
\usepackage[newcommands]{ragged2e}

% Com ragged2e e a opção "newcommands", textos curtos não-justificados
% podem gerar warnings sobre "underfull \hbox". Não há razão para pensar
% muito nesses warnings, então melhor desabilitá-los.
% https://tex.stackexchange.com/questions/17659/ragged2e-newcommands-option-produces-underfull-hbox-warnings
\makeatletter
\g@addto@macro{\centering}{\hbadness=\@M}
\g@addto@macro{\Centering}{\hbadness=\@M}
\g@addto@macro{\raggedright}{\hbadness=\@M}
\g@addto@macro{\RaggedRight}{\hbadness=\@M}
\g@addto@macro{\raggedleft}{\hbadness=\@M}
\g@addto@macro{\RaggedLeft}{\hbadness=\@M}
\g@addto@macro{\center}{\hbadness=\@M}
\g@addto@macro{\Center}{\hbadness=\@M}
\g@addto@macro{\flushleft}{\hbadness=\@M}
\g@addto@macro{\FlushLeft}{\hbadness=\@M}
\g@addto@macro{\flushright}{\hbadness=\@M}
\g@addto@macro{\FlushRight}{\hbadness=\@M}
\makeatother

% LaTeX define os comandos "MakeUppercase" e "MakeLowercase", mas eles têm
% algumas limitações; esta package define os comandos MakeTextUppercase e
% MakeTextLowercase que resolvem isso.
\usepackage{textcase}


%%%%%%%%%%%%%%%%%%%%%%%%%%%%%%%%%%%%%%%%%%%%%%%%%%%%%%%%%%%%%%%%%%%%%%%%%%%%%%%%
%%%%%%%%%%%%%%%%%%%%%%%%%%%%%%%%%%% FONTE %%%%%%%%%%%%%%%%%%%%%%%%%%%%%%%%%%%%%%
%%%%%%%%%%%%%%%%%%%%%%%%%%%%%%%%%%%%%%%%%%%%%%%%%%%%%%%%%%%%%%%%%%%%%%%%%%%%%%%%

% LaTeX normalmente usa quatro tipos de fonte:
%
% * uma fonte serifada, para o corpo do texto;
% * uma fonte com design similar à anterior para modo matemático;
% * uma fonte sem serifa, para títulos ou "entidades". Por exemplo, "a classe
%   \textsf{TimeManager} é responsável..." ou "chamamos \textsf{primos} os
%   números que...". Observe que em quase todos os casos desse tipo é mais
%   adequado usar negrito ou itálico;
% * uma fonte "teletype", para trechos de programas.
%
% A escolha de uma família de fontes para o documento por default seleciona as
% quatro fontes de uma vez.
%
% LaTeX usa por default a família de fontes "Computer Modern". Essas fontes
% precisaram ser re-criadas diversas vezes em formatos diferentes, então há
% diversas variantes dela. Com o fontenc OT1 (default "ruim" do LaTeX), a
% versão usada é a BlueSky Computer Modern, que é de boa qualidade, mas com os
% problemas do OT1. Com fontenc T1 (padrão deste modelo e recomendado), o
% LaTeX usa o conjunto "cm-super". Essa versão das fontes tem vantagens e
% desvantagens; em particular, às vezes o sistema usa fontes bitmap, que são
% ruins para leitura na tela. Ao longo do tempo, versões diferentes dessas
% fontes foram recomendadas como "a melhor"; atualmente, a melhor opção para
% usar a família Computer Modern é a versão "Latin Modern".
\usepackage{lmodern}

% Latin Modern não tem fontes bold + Small Caps, mas cm-super sim;
% assim, vamos ativar o suporte às fontes cm-super (sem ativá-las
% como a fonte padrão do documento) e configurar substituições
% automáticas para que a fonte Latin Modern seja substituída por
% cm-super quando o texto for bold + Small Caps.
\usepackage{fix-cm}

% É preciso incluir substituições para o encoding TS1 também por conta
% dos números oldstyle, porque para inclui-los nas fontes computer modern
% foi feita uma hack: os dígitos são declarados como sendo os números
% itálicos da fonte matemática e, portanto, estão no encoding TS1.
%
% Primeiro forçamos o LaTeX a carregar a fonte Latin Modern (ou seja, ler
% o arquivo que inclui "DeclareFontFamily") e, a seguir, definimos a
% substituição
\fontencoding{TS1}\fontfamily{lmr}\selectfont
\DeclareFontShape{TS1}{lmr}{b}{sc}{<->ssub * cmr/bx/n}{}
\DeclareFontShape{TS1}{lmr}{bx}{sc}{<->ssub * cmr/bx/n}{}

\fontencoding{T1}\fontfamily{lmr}\selectfont
\DeclareFontShape{T1}{lmr}{b}{sc}{<->ssub * cmr/bx/sc}{}
\DeclareFontShape{T1}{lmr}{bx}{sc}{<->ssub * cmr/bx/sc}{}

% Latin Modern não tem "small caps + itálico", mas tem "small caps + slanted";
% vamos definir mais uma substituição aqui.
\fontencoding{T1}\fontfamily{lmr}\selectfont % já feito acima, mas tudo bem
\DeclareFontShape{T1}{lmr}{m}{scit}{<->ssub * lmr/m/scsl}{}
\DeclareFontShape{T1}{lmr}{bx}{scit}{<->ssub * lmr/bx/scsl}{}

% Se fizermos mudanças manuais na fonte, estes comandos podem vir
% a ser úteis
%\newcommand\lmodern{%
%  \renewcommand{\oldstylenums}[1]{{\fontencoding{TS1}\selectfont ##1}}%
%  \fontfamily{lmr}\selectfont%
%}
%
%\DeclareRobustCommand\textlmodern[1]{%
%  {\lmodern #1}%
%}

% É possível mudar apenas uma das fontes. Em particular, a fonte
% teletype da família Computer Modern foi criada para simular
% as impressoras dos anos 1970/1980. Sendo assim, ela é uma fonte (1)
% com serifas e (2) de espaçamento fixo. Hoje em dia, é mais comum usar
% fontes sem serifa para representar código-fonte. Além disso, ao imprimir,
% é comum adotar fontes que não são de espaçamento fixo para fazer caber
% mais caracteres em uma linha de texto. Algumas opções de fontes para
% esse fim:
%\usepackage{newtxtt}
%\usepackage{DejaVuSansMono}
\usepackage{inconsolata}

% Ao invés da família Computer Modern, é possível usar outras como padrão.
% Uma ótima opção é a libertine, similar (mas não igual) à Times mas com
% suporte a Small Caps e outras qualidades. A fonte teletype da família
% é serifada, então é melhor definir outra; a opção "mono=false" faz
% o pacote não carregar sua própria fonte, mantendo a escolha anterior.
% A opção "nofontspec" elimina um problema de compatibilidade com algumas
% outras fontes; remova-a se você for utilizar XeLaTeX ou a package fontspec,
% mais abaixo.
\usepackage[mono=false,nofontspec]{libertine}
% A família libertine por padrão não define uma fonte matemática específica;
% uma opção que funciona bem com ela:
\usepackage[libertine]{newtxmath}
% Ativa apenas a fonte biolinum, que é a fonte sem serifa da família.
%\usepackage[nofontspec]{biolinum}

% Também é possível usar a Times como padrão; nesse caso, a fonte sem serifa
% é a Helvetica. Mas provavelmente libertine é uma opção melhor.
%\usepackage[helvratio=0.95,largesc]{newtxtext}

% gentium inclui apenas uma fonte serifada, similar a Garamond, que busca
% cobrir todos os caracteres unicode
%\usepackage{gentium}

% LaTeX normalmente funciona com fontes que foram adaptadas para ele, ou
% seja, ele não usa as fontes padrão instaladas no sistema: para usar
% uma fonte é preciso ativar o pacote correspondente, como visto acima.
% É possível escapar dessa limitação e acessar as fontes padrão do sistema
% com XeTeX ou LuaTeX. Com eles, além dos pacotes de fontes "tradicionais",
% pode-se usar o pacote fontspec para usar fontes do sistema.
%\usepackage{fontspec}
%\setmainfont{DejaVu Serif}
%\setmainfont{Charis SIL}
%\setsansfont{DejaVu Sans}
%\setsansfont{Libertinus Sans}[Scale=1.1]
%\setmonofont{DejaVu Sans Mono}

% fontspec oferece vários recursos interessantes para manipular fontes.
% Por exemplo, Garamond é uma fonte clássica; a versão EBGaramond é muito
% boa, mas não possui versões bold e bold-italic; aqui, usamos
% CormorantGaramond ou Gentium para simular a versão bold.
%\setmainfont{EBGaramond12}[
%  Numbers        = {Lining,} ,
%  Scale          = MatchLowercase ,
%  UprightFont    = *-Regular ,
%  ItalicFont     = *-Italic ,
%  BoldFont       = gentiumbasic-bold ,
%  BoldItalicFont = gentiumbasic-bolditalic ,
%%  BoldFont       = CormorantGaramond Bold ,
%%  BoldItalicFont = CormorantGaramond Bold Italic ,
%]
%
%\newfontfamily\garamond{EBGaramond12}[
%  Numbers        = {Lining,} ,
%  Scale          = MatchLowercase ,
%  UprightFont    = *-Regular ,
%  ItalicFont     = *-Italic ,
%  BoldFont       = gentiumbasic-bold ,
%  BoldItalicFont = gentiumbasic-bolditalic ,
%%  BoldFont       = CormorantGaramond Bold ,
%%  BoldItalicFont = CormorantGaramond Bold Italic ,
%]

% Crimson tem Small Caps, mas o recurso é considerado "em construção".
% Vamos utilizar Gentium para Small Caps
%\setmainfont{Crimson}[
%  Numbers           = {Lining,} ,
%  Scale             = MatchLowercase ,
%  UprightFont       = *-Roman ,
%  ItalicFont        = *-Italic ,
%  BoldFont          = *-Bold ,
%  BoldItalicFont    = *-Bold Italic ,
%  SmallCapsFont     = Gentium Plus ,
%  SmallCapsFeatures = {Letters=SmallCaps} ,
%]
%
%\newfontfamily\crimson{Crimson}[
%  Numbers           = {Lining,} ,
%  Scale             = MatchLowercase ,
%  UprightFont       = *-Roman ,
%  ItalicFont        = *-Italic ,
%  BoldFont          = *-Bold ,
%  BoldItalicFont    = *-Bold Italic ,
%  SmallCapsFont     = Gentium Plus ,
%  SmallCapsFeatures = {Letters=SmallCaps} ,
%]

% Com o pacote fontspec, também é possível usar o comando "\fontspec" para
% selecionar uma fonte temporariamente, sem alterar as fontes-padrão do
% documento.

% Tanto Small Caps quanto itálico (ou slanted) são "shapes" de uma fonte.
% Sendo assim, os comandos \scshape (ou \textsc) e \itshape (ou \textit) são
% "incompatíveis" entre si, ou seja, um cancela o outro. O que LaTeX faz é
% considerar que há um outro shape: "small caps + itálico" (ou "small caps +
% slanted"), chamado "scit" ou "scsl". Se a fonte oferece esse shape, é só
% usar \fontshape(scit}\selectfont. Mas isso é muito desconfortável, já que
% o usual seria algo como "\textsc{Algumas \textit{palavras} podem ser
% diferentes}". Esta package resolve esse problema.
\usepackage{slantsc}


%%%%%%%%%%%%%%%%%%%%%%%%%%%%%%%%%%%%%%%%%%%%%%%%%%%%%%%%%%%%%%%%%%%%%%%%%%%%%%%%
%%%%%%%%%%%%%%%%%%%%%%%%%%% APARÊNCIA/FORMATAÇÃO %%%%%%%%%%%%%%%%%%%%%%%%%%%%%%%
%%%%%%%%%%%%%%%%%%%%%%%%%%%%%%%%%%%%%%%%%%%%%%%%%%%%%%%%%%%%%%%%%%%%%%%%%%%%%%%%

% Formatação personalizada das listas "itemize", "enumerate" e
% "description", além de permitir criar novos tipos de listas
%\usepackage{paralist}

% Formatação personalizada do sumário, lista de tabelas/figuras etc.
%\usepackage{titletoc}

% Lembre-se que titlesec é incompatível com os comandos refsection
% e refsegment do pacote biblatex!
% Formatação personalizada de títulos, seções etc.
% Cabeçalhos dos títulos: negrito (bf), fonte um pouco menor (medium)
% e menos espaçamento vertical (compact)
%\usepackage[bf,medium,compact]{titlesec}

% Permite saber o número total de páginas; útil para colocar no
% rodapé algo como "página 3 de 10" com "\thepage de \pageref{LastPage}"
%\usepackage{lastpage}

% Formatação dos cabeçalhos e rodapés
\usepackage{fancyhdr}

% Sem linha separando o cabeçalho
\renewcommand{\headrulewidth}{0pt}

% A formatação dos cabeçalhos/rodapés envolve duas coisas:
% 1. Escolher qual texto deve ser impresso nas páginas pares/ímpares
%    (nome do capítulo ou seção, nome do autor etc.)
% 2. Escolher o lugar e a formatação desse texto e da numeração de páginas
%
% O lugar e a formatação são definidos com os comandos fancyhead e
% fancyfoot. "RO" significa "Right side of Odd pages"; "LE" significa
% "Left side of Even pages" etc.
%
% A escolha do texto é feita com os comandos chaptermark/sectionmark;
% os nomes "left/right/both" usados por esses comandos não fazem muito
% sentido, melhor pensar neles como palavras "mágicas". Para fazer
% mudanças não triviais aqui é necessário ler a documentação.
%
% O comando "\chaptermark\markboth" define o que fica guardado na variável
% "leftmark". Pode ser só "##1" (o nome do capítulo), pode ser
% "\thechapter. ##1" (aí inclui o número do capítulo), pode ser
% "\chaptername\ \thechapter. ##1" (aí inclui a palavra "capítulo") etc.
%
% O comando \sectionmark\markright" define o que fica guardado na variável
% "rightmark". Pode ser só "##1" (o nome da seção dentro do capítulo),
% pode ser "\thesection. ##1" (aí inclui o número da seção), pode ser
% "\sectionname\ \thesection. ##1" (aí inclui a palavra "seção") etc.

% Só olha até o nível 2 (seções), ou seja, não coloca nomes de
% subseções ou subsubseções nos cabeçalhos.
\setcounter{tocdepth}{2}

% Queremos colocar o número da página mais próximo da borda do papel (na
% horizontal). Para isso, vamos aumentar \headwidth, somando "tamanho da
% margem direita -10mm" (o número vai ficar a 10mm da borda).
%
% Observe que a package geometry define \evensidemargin, mas seu valor não
% necessariamente corresponde ao que queremos aqui (não sei bem como nem
% por que geometry define esse valor). Ao invés de usá-lo, vamos calcular
% manualmente.
%
% A distância entre a borda esquerda/interna do papel e o início do texto
% é dada por (1in + \hoffset + \oddsidemargin); a margem direita, portanto,
% é dada por (\paperwidth - (1in + \hoffset + \oddsidemargin + \textwidth)).
\dimdef{\othermargin}{\paperwidth - 1in - \hoffset - \oddsidemargin - \textwidth}
\addtolength{\headwidth}{\othermargin}
\addtolength{\headwidth}{-10mm}

\newcommand{\formataNumPaginas}{
  \fancyhead[RO]{\raisebox{8mm}{\footnotesize\bfseries\thepage}}
  \fancyhead[LE]{\raisebox{8mm}{\footnotesize\bfseries\thepage}}
}

\newcommand{\formataCabecalhosDinamicos}{
  \fancyhead[LO]{\scriptsize\MakeTextUppercase{\rightmark}}
  \fancyhead[RE]{\scriptsize\MakeTextUppercase{\leftmark}}
}

\fancypagestyle{mainmatter}{
  % Nome do capítulo no cabeçalho das páginas pares (e nas
  % ímpares quando não há seções)
  \renewcommand{\chaptermark}[1]{
    \markboth
      {\thechapter\hskip 0.3em |\hskip 0.3em ##1}
      {\thechapter\hskip 0.3em |\hskip 0.3em ##1}
  }

  % Número e nome da seção no cabeçalho das páginas ímpares
  \renewcommand{\sectionmark}[1]{
    \markright{\thesection\hskip 0.3em |\hskip 0.3em ##1}
  }

  \fancyhf{}
  \formataNumPaginas{}
  \formataCabecalhosDinamicos{}
}

\fancypagestyle{appendix}{
  \renewcommand{\chaptermark}[1]{%
    \markboth{%
      % Páginas ímpares: "Apêndice X"
      \appendixname\ \thechapter%
    }{%
      % Páginas pares: "X | nome do apêndice"
      \thechapter\hskip 0.3em |\hskip 0.3em ##1%
    }
  }

  \fancyhf{}
  \formataNumPaginas{}
  \formataCabecalhosDinamicos{}
}

% Na frontmatter e backmatter, não há números de capítulos e (em geral)
% não há subdivisões (capítulos/seções/subseções), apenas um nível.
% O correto, então, é usar o mesmo texto (nome do capítulo ou seção)
% nas páginas pares ou ímpares. Isso na verdade não está funcionando
% na frontmatter, pois os capítulos ali não definem os cabeçalhos (não
% executam "chaptermark/sectionmark"), mas "forçamos" a bibliografia
% e o índice a usarem.
\fancypagestyle{frontback}{
  \renewcommand{\chaptermark}[1]{\markboth{##1}{##1}}
  \renewcommand{\sectionmark}[1]{\markboth{##1}{##1}}

  \fancyhf{}
  \formataNumPaginas{}
  \formataCabecalhosDinamicos{}
}

% A página inicial de cada capítulo é automaticamente configurada para o estilo
% "plain", então vamos definir esse estilo (colocando o número de página no
% mesmo lugar das demais). As páginas iniciais também usam esse estilo.
\fancypagestyle{plain}{
  \fancyhf{}
  \formataNumPaginas{}
}


%%%%%%%%%%%%%%%%%%%%%%%%%%%%%%%%%%%%%%%%%%%%%%%%%%%%%%%%%%%%%%%%%%%%%%%%%%%%%%%%
%%%%%%%%%%%%%%%%%%%%%%%%%%%%% FIGURAS / FLOATS %%%%%%%%%%%%%%%%%%%%%%%%%%%%%%%%%
%%%%%%%%%%%%%%%%%%%%%%%%%%%%%%%%%%%%%%%%%%%%%%%%%%%%%%%%%%%%%%%%%%%%%%%%%%%%%%%%

% Permite importar figuras. LaTeX "tradicional" só é capaz de trabalhar com
% figuras EPS. Hoje em dia não há nenhuma boa razão para usar essa versão;
% pdfTeX, XeTeX, e LuaTeX podem usar figuras nos formatos PDF, JPG e PNG; EPS
% também pode funcionar em algumas instalações mas não é garantido, então é
% melhor evitar.
\usepackage{graphicx}

% Diretório onde estão as figuras; com isso, não é preciso colocar o caminho
% completo em \includegraphics. Na verdade, não precisa nem colocar a extensão
\graphicspath{{./figuras/}}

% Mais tipos de float e mais opções para personalização; este pacote
% também acrescenta a possibilidade de definir "H" como opção de
% posicionamento do float, que significa "aqui, incondicionalmente".
\usepackage{float}

% Por padrão, LaTeX prefere colocar floats no topo da página que
% onde eles foram definidos; vamos mudar isso. Este comando depende
% do pacote "float", carregado logo acima.
\floatplacement{table}{htbp}
\floatplacement{figure}{htbp}

% Garante que floats (tabelas e figuras) só apareçam após as seções a que
% pertencem. Por padrão, se a seção começa no meio da página, LaTeX pode
% colocar a figura no topo dessa página
\usepackage{flafter}
% Às vezes um float pode ser adiado por muitas páginas; é possível forçar
% LaTeX a imprimir todos os floats pendentes com o comando \clearpage.
% Esta package acrescenta o comando \FloatBarrier, que garante que floats
% definidos anteriormente sejam impressos e garante que floats subsequentes
% não apareçam antes desse ponto. A opção "section" faz o comando ser
% aplicado automaticamente a cada nova seção. "above" e "below" desabilitam
% a barreira quando os floats estão na mesma página.
\usepackage[section,above,below]{placeins}

% LaTeX escolhe automaticamente o "melhor" lugar para colocar cada float.
% Por padrão, ele tenta colocá-los no topo da página e depois no pé da
% página; se não tiver sucesso, vai para a página seguinte e recomeça.
% Se esse algoritmo não tiver sucesso "logo", LaTeX cria uma página só
% com floats. É possível modificar esse comportamento com as opções de
% posicionamento: "tp", por exemplo, instrui LaTeX a não colocar floats
% no pé da página, e "htbp" o instrui para tentar "aqui" como a primeira
% opção. O pacote "float" acrescenta a opção "H", que significa "aqui,
% incondicionalmente".
%
% A escolha do "melhor" lugar leva em conta os parâmetros abaixo, mas é
% possível ignorá-los com a opção de posicionamento "!". Dado que os
% valores default não são muito bons para floats "grandes" ou documentos
% com muitos floats, é muito comum usar "!" ou "H". No entanto, modificando
% esses parâmetros o algoritmo automático tende a funcionar bem.

% Fração da página que pode ser ocupada por floats no topo. Default: 0.7
\renewcommand{\topfraction}{.85}
% Idem para documentos em colunas e floats que tomam as 2 colunas. Default: 0.7
\renewcommand{\dbltopfraction}{.66}
% Fração da página que pode ser ocupada por floats no pé. Default: 0.3
\renewcommand{\bottomfraction}{.7}
% Fração mínima da página que deve conter texto. Default: 0.2
\renewcommand{\textfraction}{.15}
% Numa página só de floats, fração mínima que deve ser ocupada. Default: 0.5
\renewcommand{\floatpagefraction}{.66}
% Idem para documentos em colunas e floats que tomam as 2 colunas. Default: 0.5
\renewcommand{\dblfloatpagefraction}{.66}
% Máximo de floats no topo da página. Default: 2
\setcounter{topnumber}{9}
% Idem para documentos em colunas e floats que tomam as 2 colunas. Default: 2
\setcounter{dbltopnumber}{9}
% Máximo de floats no pé da página. Default: 1
\setcounter{bottomnumber}{9}
% Máximo de floats por página. Default: 3
\setcounter{totalnumber}{20}

% Define o ambiente "\begin{landscape} -- \end{landscape}"; o texto entre
% esses comandos é impresso em modo paisagem, podendo se estender por várias
% páginas. A rotação não inclui os cabeçalhos e rodapés das páginas.
% O principal uso desta package é em conjunto com a package longtable: se
% você precisa mostrar uma tabela muito larga (que precisa ser impressa em
% modo paisagem) e longa (que se estende por várias páginas), use
% "\begin{landscape}" e "\begin{longtable}" em conjunto. Note que o modo
% landscape entra em ação imediatamente, ou seja, "\begin{landscape}" gera
% uma quebra de página no local em que é chamado. Na maioria dos casos, o
% que se quer não é isso, mas sim um "float paisagem"; isso é o que a
% package rotating oferece (veja abaixo).
\usepackage{pdflscape}

% Define dois novos tipos de float: sidewaystable e sidewaysfigure, que
% imprimem a figura ou tabela sozinha em uma página em modo paisagem. Além
% disso, permite girar elementos na página de diversas outras maneiras.
\usepackage[figuresright,clockwise]{rotating}

% Captions com fonte menor, indentação normal, corpo do texto
% negrito e nome do caption itálico
\usepackage[
  font=small,
  format=plain,
  labelfont=bf,up,
  textfont=it,up]{caption}

% Sub-figuras (e seus captions) - observe que existe uma package chamada
% "subfigure", mas ela é obsoleta; use esta no seu lugar.
\usepackage{subcaption}

% Permite criar imagens com texto ao redor
\usepackage{wrapfig}

% Permite incorporar um arquivo PDF como uma página adicional. Útil se
% for necessário importar uma imagem ou tabela muito grande ou ainda
% para definir uma capa personalizada.
\usepackage{pdfpages}


%%%%%%%%%%%%%%%%%%%%%%%%%%%%%%%%%%%%%%%%%%%%%%%%%%%%%%%%%%%%%%%%%%%%%%%%%%%%%%%%
%%%%%%%%%%%%%%%%%%%%%%%%%%%%%%%%%% TABELAS %%%%%%%%%%%%%%%%%%%%%%%%%%%%%%%%%%%%%
%%%%%%%%%%%%%%%%%%%%%%%%%%%%%%%%%%%%%%%%%%%%%%%%%%%%%%%%%%%%%%%%%%%%%%%%%%%%%%%%

% Tabelas simples são fáceis de fazer em LaTeX; tabelas com alguma sofisticação
% são trabalhosas, pois é difícil controlar alinhamento, largura das colunas,
% distância entre células etc. Ou seja, é muito comum que a tabela final fique
% "torta". Por isso, em muitos casos, vale mais a pena gerar a tabela em uma
% planilha, como LibreOffice calc ou excel, transformar em PDF e importar como
% figura, especialmente se você quer controlar largura/altura das células
% manualmente etc. No entanto, se você quiser fazer as tabelas em LaTeX para
% garantir a consistência com o tipo e o tamanho das fontes, é possível e o
% resultado é muito bom. Aqui há alguns pacotes que incrementam os recursos de
% tabelas do LaTeX e alguns comandos pré-prontos que podem facilitar um pouco
% seu uso.

% LaTeX por padrão não permite notas de rodapé dentro de tabelas;
% este pacote acrescenta essa funcionalidade.
\usepackage{tablefootnote}

% Estende o ambiente tabular para que, além de "l", "c", "r" para definir se uma
% coluna deve ser alinhada à esquerda, centralizada ou à direita, seja possível
% definir a largura das colunas (além de outras pequenas modificações). Isso é
% muito útil porque LaTeX não "percebe" automaticamente quando é mais
% interessante fazer uma coluna mais estreita e forçar quebras de linha nas
% células correspondentes.
\usepackage{array}

% Se você quer ter um pouco mais de controle sobre o tamanho de cada coluna da
% tabela, utilize estes tipos de coluna (criados com base nos recursos do pacote
% array). É só usar algo como M{número}, onde "número" (por exemplo, 0.4) é a
% fração de \textwidth que aquela coluna deve ocupar. "M" significa que o
% conteúdo da célula é centralizado; "L", alinhado à esquerda; "J", justificado;
% "R", alinhado à direita. Obviamente, a soma de todas as frações não pode ser
% maior que 1, senão a tabela vai ultrapassar a linha da margem.
\newcolumntype{M}[1]{>{\centering}m{#1\textwidth}}
\newcolumntype{L}[1]{>{\RaggedRight}m{#1\textwidth}}
\newcolumntype{R}[1]{>{\RaggedLeft}m{#1\textwidth}}
\newcolumntype{J}[1]{m{#1\textwidth}}

% Permite alinhar os elementos de uma coluna pelo ponto decimal
\usepackage{dcolumn}

% Define tabelas do tipo "longtable", similares a "tabular" mas que podem ser
% divididas em várias páginas. "longtable" também funciona corretamente com
% notas de rodapé. Note que, como uma longtable pode se estender por várias
% páginas, não faz sentido colocá-las em um float "table". Por conta disso,
% longtable define o comando "\caption" internamente.
\usepackage{longtable}

% Permite agregar linhas de tabelas, fazendo colunas "compridas"
\usepackage{multirow}

% Cria comando adicional para possibilitar a inserção de quebras de linha
% em uma célula de tabela, entre outros
\usepackage{makecell}

% Às vezes a tabela é muito larga e não cabe na página. Se os cabeçalhos da
% tabela é que são demasiadamente largos, uma solução é inclinar o texto das
% células do cabeçalho. Para fazer isso, use o comando "\rothead".
\renewcommand{\rothead}[2][60]{\makebox[11mm][l]{\rotatebox{#1}{\makecell[c]{#2}}}}

% Se quiser criar uma linha mais grossa no meio de uma tabela, use
% o comando "\thickhline".
\newlength\savedwidth
\newcommand\thickhline{
  \noalign{
    \global\savedwidth\arrayrulewidth
    \global\arrayrulewidth 1.5pt
  }
  \hline
  \noalign{\global\arrayrulewidth\savedwidth}
}

% Modifica (melhora) o layout default das tabelas e acrescenta os comandos
% \toprule, \bottomrule, \midrule e \cmidrule
\usepackage{booktabs}

%%%%%%%%%%%%%%%%%%%%%%%%%%%%%%%%%%%%%%%%%%%%%%%%%%%%%%%%%%%%%%%%%%%%%%%%%%%%%%%%
%%%%%%%%%%%%%%%%%%%%%%%%%%%%%%%%% ESTRUTURA %%%%%%%%%%%%%%%%%%%%%%%%%%%%%%%%%%%%
%%%%%%%%%%%%%%%%%%%%%%%%%%%%%%%%%%%%%%%%%%%%%%%%%%%%%%%%%%%%%%%%%%%%%%%%%%%%%%%%

% acrescentamos a bibliografia/indice/conteudo no Sumário, mas excluímos as
% listas de figuras e tabelas e o próprio sumário.
\usepackage[nottoc,notlot,notlof]{tocbibind}

% Cria índice remissivo. Este pacote precisa ser carregado antes de hyperref.
% A criação do índice remissivo depende de um programa auxiliar, que pode ser
% o "makeindex" (default) ou o xindy. xindy é mais poderoso e lida melhor com
% línguas diferentes e caracteres acentuados, mas índices criados com xindy não
% funcionam em conjunto com hyperref. Para contornar esse problema,
% configuramos hyperref para *não* gerar hyperlinks no índice (mais abaixo)
% e configuramos xindy para que ele gere esses hyperlinks por conta própria.
% Se preferir usar makeindex, modifique a chamada ao pacote imakeidx (aqui)
% e altere as opções do pacote hyperref.

% Cria o arquivo de configuração para xindy lidar corretamente com hyperlinks.
\begin{filecontents*}{hyperxindy.xdy}
(define-attributes ("emph"))
(markup-locref :open "\hyperpage{" :close "}" :attr "default")
(markup-locref :open "\textbf{\hyperpage{" :close "}}" :attr "textbf")
(markup-locref :open "\textit{\hyperpage{" :close "}}" :attr "textit")
(markup-locref :open "\emph{\hyperpage{" :close "}}" :attr "emph")
\end{filecontents*}

% Cria o arquivo de configuração para makeindex colocar um cabeçalho
% para cada letra do índice.
\begin{filecontents*}{mkidxhead.ist}
headings_flag 1
heading_prefix "{\\bfseries "
heading_suffix "}\\nopagebreak\n"
\end{filecontents*}

%\usepackage[xindy]{imakeidx} % usando xindy
\usepackage{imakeidx} % usando makeindex

% Por padrão, o cabeçalho das páginas do índice é feito em maiúsculas;
% vamos mudar isso e deixar fancyhdr definir a formatação
\indexsetup{
  othercode={\chaptermark{\indexname}},
}

\makeindex[
  noautomatic,
  intoc,
  % Estas opções são usadas por xindy
  % "-C utf8" ou "-M lang/latin/utf8.xdy" são truques para contornar este
  % bug, que existe em outras distribuições tambem:
  % https://bugs.launchpad.net/ubuntu/+source/xindy/+bug/1735439
  % Se "-C utf8" não funcionar, tente "-M lang/latin/utf8.xdy"
  %options=-C utf8 -M hyperxindy.xdy,
  %options=-M lang/latin/utf8.xdy -M hyperxindy.xdy,
  % Estas opções são usadas por makeindex
  options=-s mkidxhead.ist -l -L,
]


%%%%%%%%%%%%%%%%%%%%%%%%%%%%%%%%%%%%%%%%%%%%%%%%%%%%%%%%%%%%%%%%%%%%%%%%%%%%%%%%
%%%%%%%%%%%%%%%%%%%%%%%%%%%% OUTROS PACOTES ÚTEIS %%%%%%%%%%%%%%%%%%%%%%%%%%%%%%
%%%%%%%%%%%%%%%%%%%%%%%%%%%%%%%%%%%%%%%%%%%%%%%%%%%%%%%%%%%%%%%%%%%%%%%%%%%%%%%%

% Você provavelmente vai querer ler a documentação de alguns destes pacotes
% para personalizar algum aspecto do trabalho ou usar algum recurso específico.

% Trechos de texto "puro" (tabs, quebras de linha etc. não são modificados)
\usepackage{verbatim}

% Recursos adicionais para o modo matemático
% para evitar problemas de compatibilidade com algumas fontes, o pacote
% amsthm já foi carregado mais acima
\usepackage{latexsym}
\usepackage{amsmath}
\usepackage{amssymb}
\usepackage{mathtools}

% Notação bra-ket
%\usepackage{braket}

%\num \SI and \SIrange. For example, \SI{10}{\hertz} \SIrange{10}{100}{\hertz}
%\usepackage[binary-units]{siunitx}

% Citações melhores; se você pretende fazer citações de textos
% relativamente extensos, vale a pena ler a documentação. biblatex
% utiliza recursos deste pacote.
\usepackage{csquotes}

% O comando \ref por padrão mostra apenas o número do elemento a que se
% refere; assim, é preciso escrever "veja a Figura \ref{grafico}" ou
% "como visto na Seção \ref{sec:introducao}". Usando o pacote hyperref
% (carregado mais abaixo), esse número é transformado em um hiperlink.
%
% Se você quiser mudar esse comportamento, ative as packages varioref
% e cleveref e também as linhas "labelformat" e "crefname" mais abaixo.
% Nesse caso, você deve escrever apenas "veja a \ref{grafico}" ou
% "como visto na \ref{sec:introducao}" etc. e o nome do elemento será
% gerado automaticamente como hiperlink.
%
% Se, além dessa mudança, você quiser usar os recursos de varioref ou
% cleveref, mantenha as linhas labelformat comentadas e use os comandos
% \vref ou \cref, conforme sua preferência, também sem indicar o nome do
% elemento, que é inserido automaticamente. Vale lembrar que o comando
% \vref de varioref pode causar problemas com hyperref, impedindo a
% geração do PDF final.
%
% ATENÇÃO: varioref, hyperref e cleveref devem ser carregadas nessa ordem!
%\usepackage{varioref}

%\labelformat{figure}{Figura~#1}
%\labelformat{table}{Tabela~#1}
%\labelformat{equation}{Equação~#1}
%% Isto não funciona corretamente com os apêndices; o comando seguinte
%% contorna esse problema
%%\labelformat{chapter}{Capítulo~#1}
%\makeatletter
%\labelformat{chapter}{\@chapapp~#1}
%\makeatother
%\labelformat{section}{Seção~#1}
%\labelformat{subsection}{Seção~#1}
%\labelformat{subsubsection}{Seção~#1}

% Cria hiperlinks para capítulos, seções, \ref's, URLs etc.
\usepackage[
  unicode=true,
  plainpages=false,
  pdfpagelabels,
  colorlinks=true,
  %citecolor=black,
  %linkcolor=black,
  %urlcolor=black,
  %filecolor=black,
  citecolor=DarkGreen,
  linkcolor=NavyBlue,
  urlcolor=DarkRed,
  filecolor=green,
  bookmarksopen=true,
  % hyperref não gera hyperlinks corretos em índices remissivos criados com
  % xindy; assim, desabilitamos essa função aqui e geramos os hyperlinks
  % com uma configuração especial de xindy (mais acima). Se você preferir
  % usar makeindex, (removendo a opção "xindy" do pacote imakeidx), quem
  % precisa criar os hyperlinks é hyperref. Nesse caso, desabilite a
  % próxima linha para criar hyperlinks no índice.
  %hyperindex=false,
]{hyperref}

%\usepackage[nameinlink,noabbrev,capitalise]{cleveref}
%% cleveref não tem tradução para o português
%\crefname{figure}{Figura}{Figuras}
%\crefname{table}{Tabela}{Tabelas}
%\crefname{chapter}{Capítulo}{Capítulos}
%\crefname{section}{Seção}{Seções}
%\crefname{subsection}{Seção}{Seções}
%\crefname{subsubsection}{Seção}{Seções}
%\crefname{appendix}{Apêndice}{Apêndices}
%\crefname{subappendix}{Apêndice}{Apêndices}
%\crefname{subsubappendix}{Apêndice}{Apêndices}

% ao criar uma referência hyperref para um float, a referência aponta
% para o final do caption do float, o que não é muito bom. Este pacote
% faz a referência apontar para o início do float (é possível personalizar
% também).
\usepackage[all]{hypcap}

% XMP (eXtensible Metadata Platform) is a mechanism proposed by Adobe for
% embedding document metadata within the document itself. The package
% integrates seamlessly with hyperref and requires virtually no modifications
% to documents that already exploit hyperref's mechanisms for specifying PDF
% metadata.
\usepackage{hyperxmp}

\usepackage{url}
% URL com fonte sem serifa ao invés de teletype
\urlstyle{sf}

% Permite inserir comentários, muito bom durante a escrita do texto
%\usepackage{todonotes}

% para formatar código-fonte (ex. em Java).
\usepackage{listings}

% O pacote listings não lida bem com acentos! No caso dos caracteres acentuados
% usados em português é possível contornar o problema com a tabela abaixo.
% From https://en.wikibooks.org/wiki/LaTeX/Source_Code_Listings#Encoding_issue
\lstset{literate=
  {á}{{\'a}}1 {é}{{\'e}}1 {í}{{\'i}}1 {ó}{{\'o}}1 {ú}{{\'u}}1
  {Á}{{\'A}}1 {É}{{\'E}}1 {Í}{{\'I}}1 {Ó}{{\'O}}1 {Ú}{{\'U}}1
  {à}{{\`a}}1 {è}{{\`e}}1 {ì}{{\`i}}1 {ò}{{\`o}}1 {ù}{{\`u}}1
  {À}{{\`A}}1 {È}{{\'E}}1 {Ì}{{\`I}}1 {Ò}{{\`O}}1 {Ù}{{\`U}}1
  {ä}{{\"a}}1 {ë}{{\"e}}1 {ï}{{\"i}}1 {ö}{{\"o}}1 {ü}{{\"u}}1
  {Ä}{{\"A}}1 {Ë}{{\"E}}1 {Ï}{{\"I}}1 {Ö}{{\"O}}1 {Ü}{{\"U}}1
  {â}{{\^a}}1 {ê}{{\^e}}1 {î}{{\^i}}1 {ô}{{\^o}}1 {û}{{\^u}}1
  {Â}{{\^A}}1 {Ê}{{\^E}}1 {Î}{{\^I}}1 {Ô}{{\^O}}1 {Û}{{\^U}}1
  {œ}{{\oe}}1 {Œ}{{\OE}}1 {æ}{{\ae}}1 {Æ}{{\AE}}1 {ß}{{\ss}}1
  {ű}{{\H{u}}}1 {Ű}{{\H{U}}}1 {ő}{{\H{o}}}1 {Ő}{{\H{O}}}1
  {ç}{{\c c}}1 {Ç}{{\c C}}1 {ø}{{\o}}1 {å}{{\r a}}1 {Å}{{\r A}}1
  {€}{{\euro}}1 {£}{{\pounds}}1 {«}{{\guillemotleft}}1
  {»}{{\guillemotright}}1 {ñ}{{\~n}}1 {Ñ}{{\~N}}1 {¿}{{?`}}1
}

% Opções default para o pacote listings
% Ref: http://en.wikibooks.org/wiki/LaTeX/Packages/Listings
\lstset{
  basicstyle=\footnotesize\ttfamily,  % the font that is used for the code
  numbers=left,                       % where to put the line-numbers
  numberstyle=\footnotesize\ttfamily, % the font that is used for the line-numbers
  stepnumber=1,                       % the step between two line-numbers. If it's 1 each line will be numbered
  numbersep=20pt,                     % how far the line-numbers are from the code
  commentstyle=\color{Brown}\upshape,
  stringstyle=\color{black},
  identifierstyle=\color{DarkBlue},
  keywordstyle=\color{cyan},
  showspaces=false,                   % show spaces adding particular underscores
  showstringspaces=false,             % underline spaces within strings
  showtabs=false,                     % show tabs within strings adding particular underscores
  frame=single,                       % adds a frame around the code
  framerule=0.6pt,
  tabsize=2,                          % sets default tabsize to 2 spaces
  captionpos=b,                       % sets the caption-position to bottom
  breaklines=true,                    % sets automatic line breaking
  breakatwhitespace=false,            % sets if automatic breaks should only happen at whitespace
  escapeinside={\%*}{*)},             % if you want to add a comment within your code
  backgroundcolor=\color[rgb]{1.0,1.0,1.0}, % choose the background color.
  rulecolor=\color{darkgray},
  extendedchars=true,
  inputencoding=utf8,
  xleftmargin=30pt,
  xrightmargin=10pt,
  framexleftmargin=25pt,
  framexrightmargin=5pt,
  framesep=5pt,
}

% Um exemplo de estilo personalizado para listings (tabulações maiores)
\lstdefinestyle{wider} {
  tabsize = 4,
  numbersep=15pt,
  xleftmargin=25pt,
  framexleftmargin=20pt,
}

% Outro exemplo de estilo personalizado para listings (sem cores)
\lstdefinestyle{nocolor} {
  commentstyle=\color{darkgray}\upshape,
  stringstyle=\color{black},
  identifierstyle=\color{black},
  keywordstyle=\color{black},
}

% Um exemplo de definição de linguagem para listings (XML)
\lstdefinelanguage{XML}{
  morecomment=[s]{<!--}{-->},
  morecomment=[s]{<!-- }{ -->},
  morecomment=[n]{<!--}{-->},
  morecomment=[n]{<!-- }{ -->},
  morestring=[b]",
  morestring=[s]{>}{<},
  morecomment=[s]{<?}{?>},
  morekeywords={xmlns,version,type}% list your attributes here
}

% Símbolos adicionais, como \celsius, \ohm, \perthousand etc.
%\usepackage{gensymb}

% Símbolos adicionais, como \textrightarrow, \texteuro etc.
\usepackage{textcomp}

% Permite criar listas como glossários, listas de abreviaturas etc.
% https://en.wikibooks.org/wiki/LaTeX/Glossary
%\usepackage{glossaries}

% Permite formatar texto em colunas
\usepackage{multicol}

% Os comandos \TeX e \LaTeX são nativos do LaTeX; esta package acrescenta os
% comandos \XeLaTeX e \LuaLaTeX. Você provavelmente não precisa desse recurso
% e, portanto, pode removê-la.
\usepackage{metalogo}

% O formato padrão do pacote epigraph é bem feinho...
% Outra opção para epígrafes é o pacote quotchap
\usepackage{epigraph}
\newcommand{\epigrafe}[2] {%
  \setlength{\epigraphrule}{0pt}
  \ifthenelse{\equal{}{#2}}{
    \epigraph{\itshape\RaggedLeft #1}{}
  }{
    \epigraph{\itshape\RaggedLeft #1}{--- #2}
  }
}

% https://tex.stackexchange.com/questions/22980/sentence-case-for-titles-in-biblatex
\newrobustcmd{\NoChangeOrSentenceCase}[1]{%
  \ifthenelse{\ifcurrentfield{booktitle}\OR\ifcurrentfield{booksubtitle}%
    \OR\ifcurrentfield{maintitle}\OR\ifcurrentfield{mainsubtitle}%
    \OR\ifcurrentfield{journaltitle}\OR\ifcurrentfield{journalsubtitle}%
    \OR\ifcurrentfield{issuetitle}\OR\ifcurrentfield{issuesubtitle}%
    \OR\ifentrytype{book}\OR\ifentrytype{mvbook}\OR\ifentrytype{bookinbook}%
    \OR\ifentrytype{booklet}\OR\ifentrytype{suppbook}%
    \OR\ifentrytype{collection}\OR\ifentrytype{mvcollection}%
    \OR\ifentrytype{suppcollection}\OR\ifentrytype{manual}%
    \OR\ifentrytype{periodical}\OR\ifentrytype{suppperiodical}%
    \OR\ifentrytype{proceedings}\OR\ifentrytype{mvproceedings}%
    \OR\ifentrytype{reference}\OR\ifentrytype{mvreference}%
    \OR\ifentrytype{report}\OR\ifentrytype{thesis}%
    \OR\ifentrytype{online}\OR\ifentrytype{misc}}
    {#1}
    {\MakeSentenceCase*{#1}}}


%
%tiks stuff
%

\usepackage{tikz}
\definecolor{blue}{RGB}{159, 192, 176}
\definecolor{green}{RGB}{160, 227, 127}
\definecolor{orange}{RGB}{243, 188, 125}
\definecolor{red}{RGB}{253, 123, 84}
\definecolor{nephritis}{RGB}{39, 174, 96}
\definecolor{emerald}{RGB}{46, 204, 113}
\definecolor{turquoise}{RGB}{39, 174, 96}
\definecolor{green-sea}{RGB}{22, 160, 133}

% Tikzstyles for Computation Graphs

% nodes
\tikzstyle{noop} = [circle, draw=none, fill=red, minimum size = 10pt]
\tikzstyle{op} = [circle, draw=red, line width=1.5pt, fill=red!70, text=black, text centered, font=\bf \normalsize, minimum size = 25pt]
\tikzstyle{state} = [circle, draw=blue, line width=1.5pt, fill=blue!70, text=black, text centered, font=\bf \normalsize, minimum size = 25pt]
% \tikzstyle{gradient} = [circle, draw=green, line width=1.5pt, fill=green!60, text=black, text centered, font=\bf \normalsize, minimum size = 25pt]
\tikzstyle{gradient} = [circle, draw=nephritis, line width=1.5pt, fill=nephritis!60, text=black, text centered, font=\bf \normalsize, minimum size = 25pt]
\tikzstyle{textonly} = [draw=none, fill=none, text centered, font=\bf \normalsize]

% edges
% \tikzstyle{tedge}  = [draw, thick, >=stealth, ->]
\tikzstyle{tedge}  = [draw, thick, >=latex, ->]

% namedscope
\tikzstyle{namedscope} = [circle, draw=orange, line width=1.5pt, fill=orange!60, align=center, inner sep=0pt]

% \tikzstyle{container} = [draw=none, rectangle, dotted, inner ysep=1.5em]
% \tikzstyle{novertex} = [draw=none, fill=none, text centered]
% \tikzstyle{predicate} = [ellipse, draw, thick, text centered, rounded corners, minimum size=30pt]
% \tikzstyle{aux} = [rectangle, draw, thick, text centered, rounded corners, minimum size=30pt]
% \tikzstyle{ledge}  = [draw, dashed, thick, >=stealth, ->]
% \tikzstyle{pedge}  = [draw, thick, >=stealth, ->]


\newcommand{\vect}[1]{\bm{#1}}
\newcommand{\myprime}[1]{{#1}^{\prime}}
\newcommand{\grad}[2]{\nabla_{#1} {#2}}
\newcommand{\dotp}[2]{{#1}^{\top}{#2}}
\newcommand{\dotpPright}[2]{{#1}^{\top}\left({#2}\right)}
\newcommand{\outerp}[2]{\left({#1}\right){#2}^{\top}}
\newcommand{\Jacobian}[2]{\frac{\partial #1}{\partial #2}}
\newcommand{\Vocab}{\mathbb{V}}
\DeclareMathOperator*{\argmin}{arg\,min}
\DeclareMathOperator*{\argmax}{arg\,max}
\DeclareMathOperator{\E}{\mathbb{E}}

\usepackage{lmodern}
\usepackage{booktabs}
\usepackage{bm}
\usepackage[scale=2]{ccicons}
\usepackage{pgfplots}
\usepackage{colortbl,xcolor}
\usepgfplotslibrary{dateplot}
\usepackage{setspace}
\usepackage{etoolbox}
\usepackage{xspace}
\usetikzlibrary{shapes,arrows,positioning,fit,backgrounds}
\usepackage{tkz-euclide}


    
