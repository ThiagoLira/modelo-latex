%% ------------------------------------------------------------------------- %%
\chapter{Introdução}
\label{cap:introducao}


A predição da resistência compressiva do cimento (CCS) é reconhecida como de
extrema importância para a indústria da construção civil, pois essa
capacidade é um dos fatores para avaliar a qualidade do cimento
\citep{cementml}. O CCS após 28 dias da produção do cimento (RC28) é escolhido
 como um parâmetro regulatório em normas de construção. Dado que os custos
 anuais de manutenção de estruturas são da ordem de bilhões de dólares, métodos
 estatísticos para a predição dessa quantidade ainda na fase de fabricação ganham ainda mais importância para o
 funcionamento mais eficiente da indústria. Nota-se também que essa  
 indústria representa grande parte das emissões de $CO_2$ na atmosfera \citep{cementroadmap}.

Esse trabalho é uma colaboração entre a empresa Intercement, o Laboratório de
Lógica, Inteligência Artificial e Métodos Formais do IME-USP e o Departamento de
Construção Civil da POLI-USP. Foram apresentados
10 anos de dados de diversas etapas da produção de cimento do complexo de
Cajati, uma das plantas da empresa. Esse trabalho estuda a possibilidade do uso
de modelos computacionais e estatísticos modernos em dados produzidos diariamente nas fábricas,
para a tomada de decisões na operação da fábrica.


\section {Motivação}



Trabalhos recentes aplicam técnicas estatísticas para predição de RC28 em amostras de cimento num
contexto laboratorial \citep{cementlin,nncement}.
Nosso objetivo é estender a metodologia para
levar essas análises da operação da fábrica, com dados colhidos na operação diária que já
possam entregar resultados imediatos para a ação dos operadores.

O trabalho \citep{dynstat} propõe uma metodologia de aplicação de técnicas de
regressão linear  para predição de RC28 cujos parâmetros são recalculados
sempre que um novo valor RC28 é amostrado pelo laboratório. Nossa contribuição é
similar, por meio de um tratamento completo dos dados providenciados na operação da fábrica como
\textbf{séries temporais} i.e. dados indexados por anotações de tempo.

Por muitas décadas o estado da arte desse tipo de análise temporal centrou-se
nos processos ARIMA, com a metodologia Box-Jenkins \citep{arima}.
Esses modelos têm uma excelente capacidade estatística e interpretabilidade.
Mas essa família de modelos possui suas limitações:
ela falha em modelar relações não-lineares nos dados \citep{forecasting} e 
muitos processos reais tendem a ser não-lineares por natureza.
Esses modelos também têm dificuldade de incorporar variáveis independentes (i.e.
além da própria série alvo, RC28 no caso desse trabalho) à série temporal nas predições \citep{ubertime}.


A modelagem de séries temporais tem se beneficiado pelo desenvolvimento de novos
métodos de Aprendizado de Máquina (ML), e mais recentemente de Aprendizado Profundo (DL).
Modelos de DL diferem de abordagens estatísticas clássicas por serem orientados
a dados \citep{dlbook}.
Esses modelos não necessitam de seleção de atributos, ou seja, é minimizada a
necessidade de especialistas para a escolha de variáveis, e esse tipo de modelo também não
assumem muito sobre a distribuição geradora dos dados, como por exemplo a
família ARIMA, que requisita do engenheiro de dados a especificação da
sazonalidade e estacionariedade dos dados \citep{arima}.
Redes Neurais são capazes de aprender sozinhas diversas características dos dados sem necessidade de intervenção. 
Alguns trabalhos uniram a flexibilidade de redes neurais com a robustez de
métodos ARIMA \citep{DIAZROBLES20088331,KHASHEI2010479},
criando soluções melhores que qualquer um desses modelos sozinho. Porém, uma das
arquiteturas de DL mais presentes atualmente em análise de séries temporais
são as Redes Neurais Recorrentes. Esses modelos já foram usados com
melhores resultados do que métodos clássicos para prever séries temporais de
demanda de energia elétrica e de carros em um aplicativo de corridas popular
\citep{energylstm,ubertime}. Outro benefício de modelos baseados em redes
neurais são a capacidade de modelar não-linearidades complexas e também de não
assumir independência estatística entre as variáveis de entrada \citep{dlbook}.


Um método de estimação adequado às incertezas do processo deve ser capaz de prever o sensível nível de imprecisão e
incerteza que ocorre no processo industrial,
refletindo assim as condições reais observadas na fábrica. A estatística bayesiana oferece técnicas
mais robustas para a modelagem de incerteza \citep{bayesml}. E num contexto em
que possuímos muitos dados, técnicas de DL também podem ser aplicadas com um
desenvolvimento bayesiano para a propagação de erro e cálculo de incertezas
\citep{ubertime,Gal2016Uncertainty}. Outro benefício de técnicas propostas
recentemente na área de predição de séries temporais com DL consiste no uso de diversas séries temporais similares para o treinamento de uma
única tarefa, e.g. históricos de consumo de energia elétrica de milhares
de residências diferentes. Métodos clássicos como ARIMA têm 
dificuldade na incorporação de diversas séries temporais para a resolução de uma
só tarefa, sendo necessário que os modelos sejam treinados individualmente para
cada série temporal e então agregados para a tomada de decisões. Trabalhos como
\citep{ubertime,deepar,deepfactors} propõe métodos robustos e com parâmetros
compartilhados (i.e. globais) no uso de um número arbitrário de séries temporais
para uma mesma tarefa.



%% ------------------------------------------------------------------------- %%
\section{Produção de Cimento}
\label{sec:producao}

Nessa seção a produção de cimento será brevemente descrita, usando como base a Figura~\ref{fig:cimento}:  

\begin{figure}[H]
\centering
\includegraphics[width=0.9\columnwidth]{cimento.png}
\caption{Representação das Diversas Etapas da produção de Cimento \citep{cementroadmap}}
\label{fig:cimento}
\end{figure}


As etapas de produção serão explicadas por número como indicado na imagem: \\

\begin{itemize}

\item[1] Depósitos ricos em carbonato de cálcio são minerados para extração desse químico. Normalmente a planta é próxima da mina.
\item[2] O material extraído é triturado em pedaços de até 10cm de diâmetro. Diferentes materiais são misturados ao resultado da tritura, de modo a manter a composição química desejada. 
\item[3] A farinha crua é pré-aquecida para que depois no forno as reações químicas aconteçam mais rápido. 
\item[4] O carbonato de cálcio é transformado em CaO por meio de reações químicas.  
\item[5] O material é introduzido ao forno, atingindo temperaturas de até $1450^\circ$C, transformando a farinha em clínquer. O clínquer é resfriado após a saída do forno. 
\item[6] O clínquer então é misturado com outros componentes que formam o cimento.
\item[7] A mistura é então moída.
\item[8] O material é empacotado, estocado e eventualmente expedido para entrega.

\end{itemize}

Os dados de interesse nesse projeto são medições de diversos parâmetros em meio ao processo de fabricação do cimento. Eles são divididos em diversas planilhas para diferentes etapas da produção de cimento, são elas, em ordem no processo:

\begin{itemize}
        \item Cimento Cru
        \item Farinha
        \item Clinquér
        \item Produção de Cimento
        \item Expedição de Cimento (Saco e Granel)
\end{itemize}

Dado que não existe uma maneira trivial de combinar dados de partes diferentes do processo (é muito difícil acompanhar o mesmo lote de cimento/matéria prima em partes diferentes do processo), as análises serão feitas em cima dos dados de Expedição de Cimento.

\subsection{Índice de Resistência de Cimento Portland}
\label{sec:rc}
A propriedade que iremos tentar prever usando nossos modelos será a resistência
compressiva do cimento. A resistência de uma amostra de cimento é ensaiada em
intervalos determinados de dias após a sua confecção. Essas informações são
anotadas nos dados com os nomes de $RCn$, com $n$ sendo a idade da amostra em
dias. Esses índices possuem a unidade de MPa, i.e. quilopascal. Valores usuais
desse ensaio são RC1, RC3, RC7 e RC28 (o mais relevante para as normas).

%% ------------------------------------------------------------------------- %%
\section{Objetivos}
\label{sec:objetivo}

O problema de negócio da produção de Cimento envolve poder decidir o mais rápido
possível e sem gastos desnecessários com experimentos e matéria prima o cimento
que será vendido. Para isso é necessário estimar o resultado do ensaio de RC28 o
quanto antes. Esse trabalho busca usar as técnicas mais recentes de Deep
Learning Bayesiano para a geração de predições probabilísticas com incertezas
calibradas a ponto de ser possível tomar desições baseadas em dados no chão de
fábrica. Substituindo então o uso de modelos empíricos e opiniões de
especialistas para cada lote que é produzido.


%% ------------------------------------------------------------------------- %%
\section{Organização do Trabalho}
\label{sec:organizacao_trabalho}

No Capítulo~\ref{cap:conceitos}, apresentamos os conceitos de Aprendizado de
Máquina, estatística frequentista e bayesiana, estimadores, métricas de acurária
e uma breve explicação de cada modelo usado. É feita também uma introdução à inferência
bayesiana, e uma possível aproximação numérica dessa técnica. No
Capítulo~\ref{cap:estudodados} são apresentadas características gerais dos
dados e testes estatísticos sobre propriedades desses. No
Capítulo~\ref{cap:resultados} enumerados as transformações necessárias para usar
os dados em modelos sequenciais e resultados de predições e métricas de
acurária de todos os modelos. No Capítulo~\ref{cap:conclusoes} são apresentadas
direções para pesquisa futura e palavras finais. 



%%% Local Variables:
%%% mode: latex
%%% TeX-master: "../quali"
%%% End: