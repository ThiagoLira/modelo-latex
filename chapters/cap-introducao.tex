%% ------------------------------------------------------------------------- %%
\chapter{Introdução}
\label{cap:introducao}


A durabilidade e vida útil do cimento tem sido o problema mais importante
enfrentado pela indústria de construção civil nas ultimas décadas
\citep{cementml}. Os custos de manutenção são da ordem de bilhões de dólares \citep{cementnn1}, e portanto, a capacidade da previsão de propriedades do cimento desde a sua produção ganha grande importância. Uni-se a isso a presença cada vez maior de métodos de ML em domínios diversos da engenharia e ciência e surge então a tentativa de usar algoritmos de aprendizado também para esse domínio (\cite{cementnn1} \cite{cementnn2}). \\ 

Esse trabalho surge de uma colaboração entre a empresa Intercement e o Laboratório de Lógica, Inteligência Artificial e Métodos Formais do IME-USP. Foram concedidos 10 anos de dados de diversas etapas da prodsução de cimento do complexo de Cajati, uma das plantas da empresa. Esse trabalho é um estudo com a análise desses dados, desde a sua limpeza até uma criação de modelos preditivos. \\


Os dados são medições de diversos parâmetros em meio ao processo do fabricação do cimento. Eles são divididos em diversas planilhas para diferentes etapas da produção de cimento, são elas, em ordem no processo:

\begin{itemize}
        \item Cimento Cru
        \item Farinha
        \item Clinquér
        \item Produção de Cimento
        \item Expedição
\end{itemize}


Esse trabalho irá aplicar métodos modernos de ML e DP para a modelagem desses dados. Dado que o problema em questão é de \textbf{regressão}, i.e. queremos prever valores numéricos, saímos do domínio canônico onde são aplicados métodos de DL, como por por exemplo NLP ou visão computacional. Nesse trabalho serão reproduzidos modelos propostos por \citet{ubertime} e \citet{energylstm} para problemas de regressão.



%% ------------------------------------------------------------------------- %%
\section{Produção de Cimento}
\label{sec:producao}

Nessa sessão será feito um breve resumo da produção de cimento, usando como base a figura a seguir: \\ 

\begin{figure}[H]
\centering
\includegraphics[width=0.9\columnwidth]{cimento.png}
\caption{Representação das Diversas Etapas da produção de Cimento \citep{cementroadmap}}
\end{figure}


As etapas de produção serão explicadas por número como indicado na imagem: \\

\begin{itemize}

\item[1] Depósitos ricos em $CaCO_3$ são mineirados para extração desse químico. Normalmente a planta é próxima da mina.
\item[2] O material extraido é triturado em pedaços de até 10cm. Diferentes materiais são misturados ao resultado da tritura, de modo a manter a composição química desejada. 
\item[3] A farinha crua é pré-aquecida para que depois no forno as reações químicas aconteçam mais rápido. 
\item[4] O cálcio é transformado em cao por meio de reações químicas.  
\item[5] O material é introduzido ao forno, atingindo temperaturas de até $1450^\circ$C, transformando a farinha em clínquer. O clínquer é resfriado após a saída do forno. 
\item[6] O clínquer então é misturado com outros componentes que formam o cimento.
\item[7] A mistura é então moída.
\item[8] O material é empacotado, estocado e eventualmente expedido para entrega.

\end{itemize}


%% ------------------------------------------------------------------------- %%
\section{Objetivos}
\label{sec:objetivo}

Esse trabalho tenta ir além de publicações recentes como \citet{cementnn1},
\citet{cementnn2} e \citet{cementml} usando métodos de DL que tem
obtido excelentes resultados em tarefas de
regressão com dados sequenciais. Esse trabalho irá reproduzir modelos
apresentados e usados em \cite{ubertime}, \cite{lstmbr} e \cite{energylstm}), onde
técnicas de DL são usadas para modelagem de dados de demanda de carros e de energia elétrica. 

%% ------------------------------------------------------------------------- %%
\section{Organização do Trabalho}
\label{sec:organizacao_trabalho}

No Capítulo~\ref{cap:conceitos}, apresentamos os conceitos de Aprendizado de
Máquina, estatística frequentista e bayesiana, além de uma breve explicação
de cada modelo usado. 



%%% Local Variables:
%%% mode: latex
%%% TeX-master: "../quali"
%%% End: