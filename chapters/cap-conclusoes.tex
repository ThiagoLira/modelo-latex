%% ------------------------------------------------------------------------- %%
\chapter{Conclusões}
\label{cap:conclusoes}


Pelos resultados obtidos, é possível concluir que não houveram ganhos significativos em usarmos modelos sequenciais de Deep Learning na modelagem dos dados. Como uma simples regressão linear consegue em alguns casos até performar melhor que uma rede neural ou uma rede recorrente, podemos afirmar também que a natureza temporal dos dados não é útil na modelagem dos mesmos. A performance de modelos que consideram esses dados não-sequenciais obteve uma performance equivalente ou melhor.

Os dados possuem uma grande quantidade de ruído proveniente de incertezas no instante das medições, o que dificulta o aprendizado seja de qual modelo for usado. E também é claro que ao longo dos 10 anos de dados possuimos mudanças bruscas de valores provenientes de fatores não presentes nos dados. Ou seja, do domínio que queremos aprender temos dados com um ruído alto e variável além de não possuirmos toda a informação do suposto "processo gerador de dados" (a distribuição de probabilidades $p$ que queremos estimar), o que dificulta qualquer modelo estatístico para o qual iremos dar essa tarefa.

Como uma regressão linear consegue resultados comparáveis a modelos diferentes e mais complexos, pode-se afirmar que a distribuição de probabilidade que queremos estimar para os preditores não é muito complexa, mas os fatores mencionados no parágrafo anterior atrapalham uma melhor performance dos modelos usados. Resultados melhores possivelmente podem ser obtidos usando métodos clássicos de Machine Learning como \textbf{feature engineering}, ou seja, aplicar um conhecimento específico do domínio do problema em questão para deixar menos para os modelos decidirem sozinhos. O que é imcompatível com métodos de Deep Learning. 




\section{Cronograma de Pesquisa}


\begin{figure}
\begin{center}

\begin{ganttchart}[y unit title=0.4cm,
y unit chart=0.5cm,
vgrid,hgrid, 
title label anchor/.style={below=-1.6ex},
title left shift=.05,
title right shift=-.05,
title height=1,
bar/.style={fill=gray!50},
incomplete/.style={fill=white},
progress label text={},
bar height=0.7,
group right shift=0,
group top shift=.6,
group height=.3,
group peaks={}{}{.2}]{24}
%labels
\gantttitle{Week}{24} \\
\gantttitle{Monday}{4} 
\gantttitle{Tuesday}{4} 
\gantttitle{Wednesday}{4} 
\gantttitle{Thursday}{4} 
\gantttitle{Friday}{4} 
\gantttitle{Saturday}{4} \\
%tasks
\ganttbar{first task}{1}{2} \\
\ganttbar{task 2}{3}{8} \\
\ganttbar{task 3}{9}{10} \\
\ganttbar{task 4}{11}{15} \\
\ganttbar[progress=33]{task 5}{20}{22} \\
\ganttbar{task 6}{18}{19} \\
\ganttbar{task 7}{16}{18} \\
\ganttbar[progress=0]{task 8}{21}{24}

%relations 
\ganttlink{elem0}{elem1} 
\ganttlink{elem0}{elem3} 
\ganttlink{elem1}{elem2} 
\ganttlink{elem3}{elem4} 
\ganttlink{elem1}{elem5} 
\ganttlink{elem3}{elem5} 
\ganttlink{elem2}{elem6} 
\ganttlink{elem3}{elem6} 
\ganttlink{elem5}{elem7} 
\end{ganttchart}
\end{center}
\caption{Gantt Chart}
\end{figure}




%------------------------------------------------------
\section{Considerações Finais} 

Texto texto texto texto texto texto texto texto texto texto texto texto texto
texto texto texto texto texto texto texto texto texto texto texto texto texto
texto texto texto texto texto texto. 

%------------------------------------------------------
\section{Sugestões para Pesquisas Futuras} 

Texto texto texto texto texto texto texto texto texto texto texto texto texto
texto texto texto texto texto texto texto texto texto texto texto texto texto
texto texto texto texto texto texto.

%%% Local Variables:
%%% mode: latex
%%% TeX-master: "../quali"
%%% bibtex-file-path: "../bibliografia"
%%% End: