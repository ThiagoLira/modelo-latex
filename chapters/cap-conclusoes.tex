%% ------------------------------------------------------------------------- %%
\chapter{Conclusões}
\label{cap:conclusoes}


Pelos resultados obtidos, é possível concluir que não houveram ganhos
significativos em usarmos modelos sequenciais de Deep Learning na modelagem dos
dados. Como uma simples regressão linear consegue em alguns casos uma
performance até melhor que uma rede neural ou uma rede recorrente, podemos afirmar também que a natureza temporal dos dados não é útil na modelagem dos mesmos. A performance de modelos que consideram esses dados não-sequenciais obteve uma performance equivalente ou melhor.

Os dados possuem uma grande quantidade de ruído proveniente de incertezas no instante das medições, o que dificulta o aprendizado seja de qual modelo for usado. E também é claro que ao longo dos 10 anos de dados possuimos mudanças bruscas de valores provenientes de fatores não presentes nos dados. Ou seja, do domínio que queremos aprender temos dados com um ruído alto e variável além de não possuirmos toda a informação do suposto "processo gerador de dados" (a distribuição de probabilidades $p$ que queremos estimar), o que dificulta qualquer modelo estatístico para o qual iremos dar essa tarefa.

Como uma regressão linear consegue resultados comparáveis a modelos diferentes e mais complexos, pode-se afirmar que a distribuição de probabilidade que queremos estimar para os preditores não é muito complexa, mas os fatores mencionados no parágrafo anterior atrapalham uma melhor performance dos modelos usados. Resultados melhores possivelmente podem ser obtidos usando métodos clássicos de Machine Learning como \textbf{feature engineering}, ou seja, aplicar um conhecimento específico do domínio do problema em questão para deixar menos para os modelos decidirem sozinhos. O que é imcompatível com métodos de Deep Learning. 




\section{Cronograma de Pesquisa}


\begin{figure}[H]
\begin{center}
     \begin{ganttchart}[%Specs
     y unit title=0.5cm,
     y unit chart=0.7cm,
     vgrid,hgrid,
     title height=1,
%     title/.style={fill=none},
     title label font=\bfseries\footnotesize,
     bar/.style={fill=blue},
     bar height=0.7,
%   progress label text={},
     group right shift=0,
     group top shift=0.7,
     group height=.3,
     group peaks width={0.2},
     inline]{1}{24}
     % labels


     % Titulo da tabela e quantos quadradinhos vai ter no total
     \gantttitle{2018-2019 - Mestrado}{24}\\

     % primeira subdivisão de quadradinhos em anos, quantos quadradinhos tem por
     % ano(precisa somar o mesmo que o valor
     % que vc definiu em cima)
    \gantttitle[]{2018}{12}                 
    \gantttitle[]{2019}{12} \\              

    %% subsubdivisão de quadradinhos em meses, tb precisa somar o total
    %% desse jeito cada quadradinho conta como 1 semana de trabalho
    %% 4 quadradinhos por mes
    \gantttitle{Outubro}{4}                
    \gantttitle{Novembro}{4}
    \gantttitle{Dezembro}{4}
    \gantttitle{Janeiro}{4}
    \gantttitle{Fevereiro}{4}
    \gantttitle{Março}{4}\\


    %%% primeiro valor nos colchetes -> nome da tarefa
    %% segundo e terceiro valores -> de qual quadradinho até qual quadradinho a
    %% tarefa vai, de acordo com o numero de quadradinhos que vc definiu la em
    %% cima
    \ganttbar[inline=false]{Tarefa 1}{1}{4}\\
    \ganttbar[inline=false]{Tarefa 2}{2}{6}\\ 
    \ganttbar[inline=false]{Tarefa 3}{3}{8} \\
    \ganttbar[inline=false]{Tarefa 4}{5}{10} \\
    \ganttbar[inline=false]{Tarefa 5}{8}{12} \\
    \ganttbar[inline=false]{Tarefa 6}{12}{20} \\
    \ganttbar[inline=false]{Tarefa 7}{13}{21} \\
    \ganttbar[inline=false]{Tarefa 8}{14}{24} \\



    
\end{ganttchart}
\end{center}
\caption{Cronograma dos próximos passos de Pesquisa}
\end{figure}

\begin{itemize}

\item[Tarefa 1: ] Refazer Experimentos sem o delay temporal, apenas com os dados
  de Expedição de Cimento
\item[Tarefa 2: ] Usar modelo para outros dados de séries temporais (e.g. clima)
\item[Tarefa 3: ] Nova compilação de resultados
\item[Tarefa 4: ] Uso de dados de outras etapas da produção de concreto simultaneamente
\item[Tarefa 5: ] Estudar alterações no modelo proposto pela Uber  
\item[Tarefa 6: ] Experimentos com mudanças propostas 
\item[Tarefa 7: ] Estudo e compilação de resultados
\item[Tarefa 8: ] Escrita da tese 


  
\end{itemize}

% ------------------------------------------------------
%% 
%% \section{Considerações Finais} 
%% 
%% Texto texto texto texto texto texto texto texto texto texto texto texto texto
%% texto texto texto texto texto texto texto texto texto texto texto texto texto
%% texto texto texto texto texto texto. 
%% 
%% %------------------------------------------------------
%% \section{Sugestões para Pesquisas Futuras} 
%% 
%% Texto texto texto texto texto texto texto texto texto texto texto texto texto
%% texto texto texto texto texto texto texto texto texto texto texto texto texto
%% texto texto texto texto texto texto.

%%% Local Variables:
%%% mode: latex
%%% TeX-master: "../quali"
%%% bibtex-file-path: "../bibliografia"
%%% End: