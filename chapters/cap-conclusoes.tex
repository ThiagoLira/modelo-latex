%% ------------------------------------------------------------------------- %%
\chapter{Conclusões}
\label{cap:conclusoes}


Após realizados os experimentos, pode-se concluir que existe um ganho de
qualidade e acurácia de predições no uso de modelos de Aprendizagem
Profunda. Em particular modelos Bayesianos são capazes de gerar predições
probabilísticas bastante versáteis para o controle de um processo industrial.

A limitação desse trabalho, porém, são a quantidade e o escopo dos dados
fornecidos pela empresa Intercement. Pela análise ter sido levantada para dados
de apenas uma fábrica os métodos possuem uma limitação intrínseca na garantia de
resultados similares em outras fábricas, ou mesmo na fábrica de Cajati em
outros períodos com anomalias diferentes das apresentadas pelos dados usados
nesse trabalho.

Não obstante, esse trabalho mostra que, mesmo com uma quantidade menor de dados
do que normalmente é possível para modelos de Aprendizado Profundo, é possível
se obter uma melhor acurácia do que com modelos clássicos, e não precisamos
abandonar as propriedades estatísticas desejáveis de modelos clássicos, pois com
novas técnicas de aprendizado automático, podemos modelar também a incerteza das predições.


\section{Sugestões para Pesquisas Futuras} 

Direções futuras de pesquisa incluem a aplicação desses modelos em dados
de fábricas diferentes, para uma possível comparação inter-fábricas. Existe
também a ideia da aplicação de \textbf{Transfer Learning}. O uso de
representações vetoriais aprendidas em um determinado conjunto de fábricas em
alguma fábrica inédita. Isso permitiria que uma quantidade enorme de dados fosse
aplicada para a criação de uma inteligência cujo domínio de aplicação seria
estendido para fábricas do mundo todo. 

Também existem os problemas técnicos para a implantação desses métodos no chão
de fábrica. É necessário realizar a integração da pipeline de treinamento e inferência ao sistema de
controle de dados das fábricas. Relacionado a isso também existe a possibilidade
da criação de uma interface gráfica para que os resultados possam ser rapidamente acessados
para um eventual uso para controle de todo o processo industrial. 


%%% Local Variables:
%%% mode: latex
%%% TeX-master: "../quali"
%%% bibtex-file-path: "../bibliografia"
%%% End: