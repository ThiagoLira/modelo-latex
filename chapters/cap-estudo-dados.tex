\chapter{Estudo dos Dados}
\label{cap:estudodados}


\section{Preparo dos Dados}


Os dados estão distribuídos em colunas com diversas propriedades analisadas em
laboratório de cada lote retirado da fábrica. Existem diversas planilhas para
lotes retirados de diversas partes do processo de fabricação. Devido a dificuldades logísticas de se casar um
mesmo lote de cimento em diferentes partes do processo (e.g. o mesmo lote ao
sair do forno e depois de finalizado e pronto para expedição), estudos serão feitos apenas para os dados de
\textit{expedição de cimento}. 

A Tabela~\ref{tb:vars} define todas as variáveis, dependentes e independentes,
que são usadas na modelagem.


\begin{table}[]
  \resizebox{\textwidth}{!}{\begin{tabular}{|l|llllll}
\cline{1-1}
\multicolumn{1}{|c|}{\textbf{ Variáveis (unidade)}}         &                                &                              &                           &                             &                               &                               \\ \hline
Composição Química (\%)                                   & \multicolumn{1}{l|}{$AL_20_3$} & \multicolumn{1}{l|}{$SIO_2$} & \multicolumn{1}{l|}{MGO}  & \multicolumn{1}{l|}{RICARB} & \multicolumn{1}{l|}{$P_2O_5$} & \multicolumn{1}{l|}{$F_2O_3$} \\ \hline
Água (\%)                                                 & \multicolumn{1}{l|}{AGP}       &                              &                           &                             &                               &                               \\ \cline{1-3}
Tempo até o começo e fim do endurecimento do material (s) & \multicolumn{1}{l|}{IP}        & \multicolumn{1}{l|}{FP}      &                           &                             &                               &                               \\ \cline{1-3}
Finura Blaine ($cm^{2}$/g)                                & \multicolumn{1}{l|}{SBL}       &                              &                           &                             &                               &                               \\ \cline{1-4}
Resistência Compressiva do Cimento (kPA)                  & \multicolumn{1}{l|}{RC3}       & \multicolumn{1}{l|}{RC7}     & \multicolumn{1}{l|}{RC28} &                             &                               &                               \\ \cline{1-4}
\end{tabular}}
\caption{Variáveis presentes nos dados de expedição de cimento cedidos pela Intercement}
\label{tb:vars}
\end{table}

Os dados de expedição de cimento são anotados aproximadamente por dia, possuindo 2520
entradas para 3650 dias distintos. Existem dias sem dados presentes. O
período contemplado pela planilha de dados vai do dia 02/01/2008 até o dia 29/12/2018.

\subsection{Dados faltantes}

Embora tenhamos uma quantidade razoável de dias com dados presentes, esses muitas vezes não possuem algum valor de alguma variável.
A seguir vemos para os dados de Expedição de Cimento, para cada uma de suas variáveis de entrada, a porcentagem de dados presentes. 


\begin{figure}[H]
  \centering
  \includegraphics[width=0.9\columnwidth]{slides_dados_pct.pdf}
  \caption{Porcentagem de dados faltantes por variável para dados de Expedição de Cimento}
  \label{fig:dadosfalta}
\end{figure}


\section{Reamostragem dos dados}

Uma dos requisitos dos dados assumidos pelos modelos de séries temporais, que os mesmos
são espaçados regularmente pelo incremento de tempo escolhido, sem lacunas.
Portanto, os dados foram modificados para que entradas anotadas no mesmo dia
sejam unificadas em um único dia. Além disso, para dias no intervalo de tempo
estudado que não possuam entradas, foram criadas entradas com valores
artificiais que não perturbem a distribuição dos dados, e.g. os valores da
última entrada válida são copiados para frente até que se possua um novo valor.

A Figura \ref{fig:reamos} mostra como os dados estavam antes da reamostragem
para que eles se tornem diários: 

\begin{figure}[H]
  \centering
  \includegraphics[width=0.5\columnwidth]{slides_dados_antes_resample.pdf}
  \caption{Distribuição das distâncias entre entradas subsequentes dos dados de Expedição de Cimento, antes da reamostragem.}
  \label{fig:reamos}
\end{figure}



\subsection{RC3, RC7 e RC28}

Assim como proposto em \cite{grecialin}, iremos usar dados supostos disponíveis diariamente no chão de fábrica. Portanto, para um lote recém expedido,
não podemos usar os índices RC3 e RC7 desse mesmo lote para auxiliar na predição
de RC28, visto que esses ainda não foram experimentados em laboratório, esses
experimentos levam respectivamente 3 e 7 dias para ficarem prontos. Dessa maneira, iremos usar os últimos índices RC3 e RC7 disponíveis para o lote expedido no dia $t$,
i.e. o índice RC3 do lote do dia $t-3$ e RC7 do lote dia $t-7$, que acabaram de ser medidos. 

A Figura~\ref{fig:gridcorr} mostra um correlograma entre os índices RC3 e RC7
atrasados e o RC28 diário. 

\begin{figure}[H]
  \centering
  \includegraphics[width=0.9\columnwidth]{corr_grid.pdf}
  \caption{Correlograma dos índices de resistência compressiva}
  \label{fig:gridcorr}
\end{figure}

A Tabela~\ref{tabelacorr} mostra o restante das correlações entre as demais
colunas de dados e o nosso objetivo, o índice RC28:


\begin{table}[H]
  \centering
\begin{tabular}{lr}
  \toprule
  {} &      RC28 \\
  \midrule
  AGP   &  0.592847 \\
  AL2O3 &  0.463414 \\
  SIO2  & -0.053178 \\
  MGO   & -0.371414 \\
  IP    & -0.132297 \\
  FP    & -0.419800 \\
  SBL   &  0.396555 \\
  PF    & -0.480720 \\
  P2O5  &  0.292252 \\
  RC28  &  1.000000 \\
  \bottomrule
\end{tabular}
\caption{Tabela mostrando correlações entre as variáveis e o alvo para o problema de aprendizado.}
\label{tabelacorr}
\end{table}


É notável que o RC28 não possui uma alta correlação com nenhuma das propriedades
do cimento. Tendo essa correlação apenas com os outros índices de resistência compressiva.




%%% Local Variables:
%%% mode: latex
%%% TeX-master: "../quali"
%%% End:
