%% ------------------------------------------------------------------------- %%
\chapter{Conceitos}
\label{cap:conceitos}


\section{Machine Learning}

O campo de Machine Learning (ML) é uma ramo da Ciência da Computação que se arma de métodos estatísticos para criar sistemas que podem aprender (i.e. melhorar sua acurácia) através de dados.

Algoritmos de ML podem ser divididos nas categorias de aprendizado supervisionado, não supervisionado e aprendizado por reforço \citep{dlbook}. Técnicas dos dois primeiros tipos serão usadas para os dados da Intercement.

\subsection{Aprendizado Supervisionado}

Aprendizado Supervisionado consiste a grosso modo em aprender uma distribuição de probabilidade do tipo $p(y | x)$, ou seja, para diversos exemplos de vetores $x$ são fornecidas anotacoes $y$, e desejamos então criar predições de anotacoes $y'$ para novas entradas $x'$. Muitos algoritmos resolvem esse problema por meio da busca pelo conjunto de parânetros $\theta$ que minimizem o erro na família de distribuições $p(y | x,\theta)$.

\subsection{Aprendizado Não Supervisionado}

Para o caso de Aprendizado Não Supervisionado, mantendo a estrutura do exemplo anterior, desejaríamos então modelar uma distribuição do tipo $p(x)$, onde temos também diversos exemplos de vetores aleatórios $x$ e podemos estar estudando alguma propriedade importante dessa distribuição.

\section{Estatística Frequentista e Estatística Bayesiana}

Como nesse trabalho serão usados métodos de inferência Bayesiana aplicados a ML, cabe então uma breve elaboração das diferenças entre os dois grandes \textit{aproacches} da estatística.\\

Digamos que exista um evento aleatório que tenha um resultado com probabilidade $p$ de acontecer. Intuitivamente, se pudessemos repetir infinitas vezes esse evento, a proporção de vezes que esse resultado irá acontecer se aproximará de arbitrariamente de $p$. E então entenderíamos a probabilidade $p$ meramente como uma proporção de resultados positivos em uma certa amostra de experimentos. Mas e se o evento não pudesse ser repetido? Quando físicos criam modelos para explicar o nascimento do universo, é impossível pensar em repetir o Big Bang infinitas vezes para que se possam estimar probabilidades de certos eventos cosmologicos acontecerem. Nesse segundo caso, resultados são derivados de \textbf{graus de certeza}, onde a chance de um evento acontecer é estimada pela aplicação de conhecimentos prévios em vista de algo que foi observado posteriormente. A primeira maneira de se entender estatística é chamada de Frequentista e a segunda de Bayesiana. \\

E no campo de ML, as duas maneiras de se gerar predições são estimadores frequentistas e inferência Bayesiana \citep{dlbook}.

\subsection{Estimação por Log-verossimilhança}
 
Um exemplo de estimação frequentista que será usada nesse trabalho é a de log-verossimilhança. 
Ela segue o princípio de maximizar a verossimilhança (i.e. a probabilidade de
termos observado $X$ dados os parâmetros $p(X | \theta)$). Ou seja, dada uma matriz de dados $X$ e um conjunto de parâmetros $\theta$, a estimador de máxima verossimilhança de $\theta$ é dado por: \\

\[ \hat{\theta} = \argmax_{\theta} p_{modelo}(X;\theta) \] 

Onde $p_{modelo}$ busca aproximar a real distribuição geradora dos dados $p$. Assumindo dados i.i.d e trocando a multiplicação por soma de logaritmos temos: \\

\[ \hat{\theta} = \argmax_{\theta} \sum_{i=1}^{m} \log p_{modelo}(x^{(i)};\theta) \]

Maximizar a equação anterior é equivalente a minimizar a entropia-cruzada entre
nossas anotações reais $y$ e as predições feitas pelo modelo $\hat{y}$
\citep{dlbook}. \\

A entropia de uma variável aleatória $X$ cuja função densidade
de probabilidade seja $p(x)$ pode ser calculada por: \\

\[ H(X)  = - \sum p(x_i)*\log p(x_i) \]

Ou ainda:

\[H(X) = - \mathop{\mathbb{E}}_p[\log p(X)] \]

Com o operador $\mathop{\mathbb{E}}_p[X]$ representando o valor esperado da
variável aleatória $X$ na distribuição $p$. \\

Sejam $p,p_{modelo}$ duas funções de distribuição de probabilidade. A entropia-cruzada de $p_{modelo}$ em $p$ é definida pelo valor
esperado da entropia de $p_{modelo}$ na distribuição $p$: \\

\[H(p,p_{modelo}) =  \mathop{\mathbb{E}}_p[\log p_{modelo}(x)] \]

Se a distribuição $p_{modelo}$ for parametrizada pelo vetor $\theta$, a equação
acima toma a forma: \\


\[H(p,p_{modelo}) =  \mathop{\mathbb{E}}_p[\log p_{modelo}(x ; \theta)] \]

Minimizar essa equação é equivalente a maximizar a equação da
log-verossimilhança. Esse será o objetivo (função de custo) dos modelos de rede neural usados
nesse trabalho.  


\subsection{Inferência Bayesiana em Machine Learning}

O tratamento Bayesiano para modelos de ML é bastante diverso dos frequentistas.
Em uma análise frequentista estima-se um valor de $\theta$ e então todas as
predições são feitas a partir desse valor. No caso Baysiano se consideram todos
os possíveis valores de $\theta$ ao se fazer uma predição. É preciso especificar
um grau de certeza \textbf{a priori} $p(\theta)$ sobre os parâmetros, e então consideramos que os dados \textbf{foram observados} e usamos a lei de Bayes para calcular a \textbf{probabilidade} posterior $p(\theta | X,Y)$ usando para tal $p(X,Y | \theta)$, a chamada verossimilhança. 

\[    p(\theta | X,Y) = \frac{p(Y| X,\theta) p(\theta)}{p(X)}   \]

Finalmente, para realizar uma inferência devemos integrar por toda a distribuição $p(\theta)$ marginalizando esse parâmetro. Se por exemplo queremos uma nova anotação $y^*$ para um novo dado $x^*$:

\[ p(y^* | x^* , X,Y) = \int  p(y^* | x^*,\theta) p(\theta | X,Y)  d\theta \]

Essa integral é geralmente intratável pela dificuldade de se calcular
analiticamente $p(\theta | X,Y)$ \citep{ubertime}, e portanto será usada uma
técnica que aproxima uma inferência Bayesiana numericamente em redes neurais, o Monte Carlo
Dropout \citep{dropbayes}. \\

O MC Dropout consiste no uso de \textbf{Dropout} em todas as camadas da rede
neural, i.e. descartar ativações aleatóriamente entre duas camadas da rede com
probabilidade $p$. Pode-se demonstrar que isso é equivalente a considerar os
parâmetros $W$ com uma distribuição a priori $W \sim \mathcal{N}(0,\,I)$. Se realizam então $B$ computações estocásticas de uma mesma
saída $y^*$, sejam elas $Y^* = \{y^*_{(1)},y^*_{(2)}, \dots , y^*_{(B)}\}$. A incerteza
  dessa medida é calculada então pela variância amostral de $Y^*$.

%% ------------------------------------------------------------------------- %%

\subsection{Temporalidade dos Dados}

Em um problema canônico de aprendizado supervisionado gostariamos de aprender
uma função $f$ tal que para um par inédito de dados $x^*,y^*$, a nossa função
dependa apenas de $x^*$ para gerar uma predição. Para os dados da Intercement pode ser que um valor i.e. índice de dureza dependa não só da última entrada, mas de diversas entradas anteriores. 

Ou seja, nossos dados podem ter sido gerados por uma distribuição de probabilidade da forma $p(y | x_{t} ,x_{t -1},x_{t -2},x_{t-3} , \dots, x_{t-T})$, onde uma saida $y$ é condicionada pelas últimas $T$ entradas. Para resolver um problema de aprendizado dessa natureza, devemos usar \textbf{modelos sequenciais}. 

Iremos então experimentar com modelos sequenciais e não-sequencias para testar a acurária de ambos no problema em questão.


\section{Modelos Usados} 

\subsubsection{Redes Neurais}


Redes neurais são aproximadores universais de funções. Se tivermos um problema
de classificação onde queremos aprender uma função da forma $y = f^*(x)$, uma
Rede Neural define um mapeamento $y = f(x ; \theta)$, onde $\theta$ é o vetor de
parâmetros que serão aprendidos com o fim de minimizar o erro do aproximador.

O modelo é uma composição de funções que unem uma transformação linear e
a aplicação de uma função não-linear $\sigma$: \\

\[ f(x)=  \sigma(W*x + b) \]

A computação da saída de uma rede neural então pode ser escrita como:

\[   y = f_n \circ f_{n-1} \circ f_{n-2} \dots f_1(x)  \]

Para uma rede neural de $n$ \textbf{camadas}. Onde cada camada será uma função
$f_i$ cujos parâmetros são $W$ e $b$. Portanto, para a i-ésima camada da rede
sua saída sera da forma: 

\[ f_i (x)=  a_i = \sigma(W_i*a_{i-1} + b_i) \]

Onde $a_{i-1}$ é a saída da camada anterior, também chamada de
\textbf{ativação}. A saída dessa camada é então a sua ativação $a_i$. \\ 

Vale notar que uma saída $y$ calculada por uma rede neural depende unicamente dos
seus parâmetros e da entrada $x$. Isso não será verdade para os modelos
\textbf{sequenciais} que também serão usados nesse trabalho, onde o estado
interno de computação desses modelos é usado como entrada para uma próxima
iteração. \\

A seguir está reproduzida uma rede neural simples com uma camada oculta e dois neurônios de saída.


%\begin{figure}[H]
\centering

\scalebox{0.7}{
\begin{tikzpicture}[auto]

% operations =============================

% input layer
\node[op] (x5) {$x_5$};
\node[op, above=2.5pt of x5] (x4) {$Fe_2O_3$};
\node[op, above=2.5pt of x4] (x3) {$CaO$};
\node[op, above=2.5pt of x3] (x2) {$SiO_2$};
\node[op, above=2.5pt of x2] (x1) {$Al_2O_3$};
\node[op, below=2.5pt of x5] (x6) {$SO_3$};
\node[op, below=2.5pt of x6] (x7) {$P_2O_5$};
\node[op, below=2.5pt of x7] (x8) {$MgO$};
\node[op, below=2.5pt of x8] (x9) {$Na_2O$};
\node[op, below=2.5pt of x9] (x10) {$K_2O$};

% hidden layer
\node[op,  right=130pt of x5] (v2) {$v_2$};
\node[op, below=2.5pt of v2] (v3) {$v_3$};
\node[op, below=2.5pt of v3] (v4) {$v_4$};
\node[op, above=2.5pt of v2] (v1) {$v_1$};

\node[op,  right=40pt of v2] (h2) {$h_2$};
\node[op, below=2.5pt of h2] (h3) {$h_3$};
\node[op, below=2.5pt of h3] (h4) {$h_4$};
\node[op, above=2.5pt of h2] (h1) {$h_1$};


% output layer
\node[op,  right=60pt of h2] (z1) {$RC3$};
\node[op,  right=60pt of h3] (z2) {$RC7$};

%\node[op,  right=50pt of z1] (y1) {$\hat{y}_1$};
%\node[op,  right=50pt of z2] (y2) {$\hat{y}_2$};


% edges input layer to hidden
\path[tedge] (x1) -- (v1);
\path[tedge] (x1) -- (v2);
\path[tedge] (x1) -- (v3);
\path[tedge] (x1) -- (v4);

\path[tedge] (x2) -- (v1);
\path[tedge] (x2) -- (v2);
\path[tedge] (x2) -- (v3);
\path[tedge] (x2) -- (v4);

\path[tedge] (x3) -- (v1);
\path[tedge] (x3) -- (v2);
\path[tedge] (x3) -- (v3);
\path[tedge] (x3) -- (v4);

\path[tedge] (x4) -- (v1);
\path[tedge] (x4) -- (v2);
\path[tedge] (x4) -- (v3);
\path[tedge] (x4) -- (v4);

\path[tedge] (x5) -- (v1);
\path[tedge] (x5) -- (v2);
\path[tedge] (x5) -- (v3);
\path[tedge] (x5) -- (v4);

\path[tedge] (x6) -- (v1);
\path[tedge] (x6) -- (v2);
\path[tedge] (x6) -- (v3);
\path[tedge] (x6) -- (v4);

\path[tedge] (x7) -- (v1);
\path[tedge] (x7) -- (v2);
\path[tedge] (x7) -- (v3);
\path[tedge] (x7) -- (v4);

\path[tedge] (x8) -- (v1);
\path[tedge] (x8) -- (v2);
\path[tedge] (x8) -- (v3);
\path[tedge] (x8) -- (v4);

\path[tedge] (x9) -- (v1);
\path[tedge] (x9) -- (v2);
\path[tedge] (x9) -- (v3);
\path[tedge] (x9) -- (v4);

\path[tedge] (x10) -- (v1);
\path[tedge] (x10) -- (v2);
\path[tedge] (x10) -- (v3);
\path[tedge] (x10) -- (v4);

% edges hidden to hidden
\path[tedge] (v1) edge node[above=1pt] {{\Large$\sigma$}}  (h1) ;
\path[tedge] (v2) edge node[above=1pt] {{\Large$\sigma$}}  (h2) ;
\path[tedge] (v3) edge node[above=1pt] {{\Large$\sigma$}}  (h3) ;
\path[tedge] (v4) edge node[above=1pt] {{\Large$\sigma$}}  (h4) ;

% edges hidden to output
\path[tedge] (h1) -- (z1);
\path[tedge] (h1) -- (z2);

\path[tedge] (h2) -- (z1);
\path[tedge] (h2) -- (z2);

\path[tedge] (h3) -- (z1);
\path[tedge] (h3) -- (z2);

\path[tedge] (h4) -- (z1);
\path[tedge] (h4) -- (z2);

% edges output to output
%\path[tedge] (z1) edge node[above=1pt] {{\Large softmax}}  (y1) ;
%\path[tedge] (z2) edge node[above=1pt] {{\Large softmax}}  (y2) ;




\end{tikzpicture}
} % scalebox
\end{figure}


Na imagem mostramos como seria uma rede que usa como parâmetros alguns dos dados de Farinha para modelar os índices RC3 e RC7 dos dados de expedição.

\bigskip
\subsubsection{Regressão Linear}
Os modelos são também comparados com uma regressão linear. Que usa estimação por mínimos quadrados para calcular um peso para cada parâmetro de entrada. De modo que a soma ponderada por esses pesos possa aproximar nosso alvo.


\subsubsection{Random Forest}

Random Forests são um método de \textbf{Ensemble Learning} para classificação ou regressão. \textbf{Ensemble Learning} é uma família de técnicas no qual diversos modelos "fracos" são usados em conjunto com algum sistema de votação para que a a acurária do sistema em conjunto se torne melhor que a de qualquer um dos modelos sozinho. Seguindo essa ideia, Random Forests são conjuntos de diveras árvores de decisão simples unidas por um meta-algoritmo de votação para que se produza uma predição muito mais eficaz.


\subsection{Modelo Sequencial}
Com o sucesso de modelos sequenciais no campo do Deep Learning, iremos averiguar se é possível modelar sequencialmente os dados da produção de cimento usando dois desses modelos. 
\\

\subsection{Rede Neural Recorrente}
% 
\scalebox{1}{
\begin{tikzpicture}[auto]

% RNN state cell =============================
\node[state] (h) {$\vect{h}$};
\node[op, below=30pt of h] (x) {$\vect{x}$};
\node[op, above=30pt of h] (yhat) {$\hat{\vect{y}}$};



% edges
\path[tedge] (x) edge node[below right= -4pt] {$\vect{U}$}  (h) ;
\path[tedge] (h) edge [out=-400,in=-320,looseness=8, distance=125pt] node[above right] {$\vect{W}$} (h);
\path[tedge] (h) edge node[below right = -4pt] {$\vect{V}$} (yhat);


\end{tikzpicture}
} % scalebox


A família das Redes Neurais Recorrentes é composta por modelos especializados
em processar dados sequenciais, da forma $x^{(1)},x^{(2)} ,x^{(3)}\dots ,x^{(T)}$. Uma rede neural recorrente é definida por uma função com
\textbf{recorrência} ou \textbf{recursão}, de modo que no processamento de uma
sequência o estado da rede seja de certo modo propagado temporalmente. A equação
a seguir ilustra como seria uma função com recorrência: \\

\[h^{(t)} = f(h^{(t-1)},x^{(t)};\theta)\]

Nessa equação vemos que o vetor $h$ na iteração $t$ tem seu valor dependente
desse mesmo vetor na última iteração. RNNs aprendem a usar esse vetores com
essas propriedades como
\say{resumos} das iterações passadas, e usam-nos ao lado da entrada e dos
parâmetros para o cálculo da saída em cada iteração.

Como podemos ver na imagem, a entrada $x$, ao lado do estado interno $W$, são usados para gerar uma predição. Essa por sua vez é comparada com o nosso dado real para que se calcule um erro. O estado $W$ é calculado em cada iteração e usado no cálculo da próxima predição. De modo que esse estado é capaz de transmitir temporalmente ao longo do treinamento informações de dados anteriores.
\\

Essa classe de modelos normalmente é usada para modelagem de linguagem. Buscando estimar uma distribuição de probabilidade $p(w_t | w_{t-1},w_{t-2},w_{t-3} \dots ) $ onde os $w_i$ são palavras subsequentes de um texto. Normalmente um modelo dessa natureza busca resolver um problema de classifição, onde a próxima palavra a ser prevista pelo modelo é uma entre todas as possibilidades de um certo vocabulário. No caso do domínio em questão desejamamos resolver um problema de regressão, onde nosso alvo é um valor numérico. Para treinar um desses modelos, precisamos usar como entrada exemplos subsequentes de dados, onde cada exemplo de entrada tem um exemplo pareado de saída. Basicamente redes neurais recorrentes funcionam recebendo um exemplo de entrada, criando uma representação interna com o mesmo e então gerando uma saída e comparando essa saída com o exemplo de saída real, gerando um erro. Finalmente, esse erro é propagado para alterar seus parâmetros (com o fim de achar um conjunto de parâmetros que gere boas previsões). Podemos vizualizar esse modelo também ao longo do tempo na imagem a seguir:


% % RNN STATE CELL ====================================
\newcommand{\rnnSimple}[4]{

% operations
\node[state, minimum size=40pt,#4] (h#3) {$\vect{h}^{#1}$};
\node[op, minimum size=40pt,below=30pt of h#3] (x#3) {$\vect{x}^{#1}$};
\node[op, minimum size=40pt, above=30pt of h#3] (yhat#3) {$\hat{\vect{y}}^{#1}$};

% edges
\path[tedge] (x#3) edge node[below right= -4pt] {$\vect{U}$} (h#3);
\path[tedge] (h#3) edge node[below right = -4pt] {$\vect{V}$} (yhat#3);
}

\begin{figure}[H]
\hspace*{-1.0cm}
\scalebox{0.9}{
\begin{tikzpicture}[auto]

% timestep 1
\rnnSimple{(1)}{(0)}{t1}{}

% % timestep 0
\node[state, minimum size=40pt,left=50pt of ht1] (ht0) {$\vect{h}^{(0)}$};

% % timestep 2
\rnnSimple{(2)}{(1)}{t2}{right=50pt of ht1};


% % timestep 2
\rnnSimple{(3)}{(1)}{t3}{right=50pt of ht2};


% % state transfers
\path[tedge] (ht0) edge node[above right = 2pt] {$\vect{W}$} (ht1);
\path[tedge] (ht1) edge node[above right = 2pt] {$\vect{W}$} (ht2);
\path[tedge] (ht2) edge node[above right = 2pt] {$\vect{W}$} (ht3);

\end{tikzpicture}
}%\scalebox
\end{figure}



Essa imagem mostra exatamente o mesmo modelo da imagem anterior, porém, agora visualizamos o modelo a cada iteração temporal. O estado W é usado como entrada juntamente com o proximo $x_i$ para uma nova iteração.

\bigskip

Como já explicado anteriormente, nossos dados de entrada e saída não estão necessariamente pareados perfeitamente dia a dia. Portanto, foi necessário achar intervalos de tempo nos dados onde existe esse pareamento. Isso reduz drasticamente quais períodos representados nos dados realmente podem ser usados para treinar um desses modelos.


\subsubsection{LSTM}

LSTMs são um tipo de RNN que por meio de sua arquitetura permitem que sequências
maiores sejam processadas sem que o fluxo dos gradientes propagados pela rede se torne
numericamente problemático (i.e. tendendo a 0 ou a infinito).

O diagrama a seguir ilustra o fluxo dos sinais dentro de uma célula de LSTM: \\

\begin{center}


\begin{tikzpicture}[
    prod/.style={circle, draw, inner sep=0pt},
    ct/.style={circle, draw, inner sep=5pt, ultra thick, minimum width=10mm},
    ft/.style={circle, draw, minimum width=8mm, inner sep=1pt},
    filter/.style={circle, draw, minimum width=7mm, inner sep=1pt, path picture={\draw[thick, rounded corners] (path picture bounding box.center)--++(65:2mm)--++(0:1mm);
    \draw[thick, rounded corners] (path picture bounding box.center)--++(245:2mm)--++(180:1mm);}},
    mylabel/.style={font=\scriptsize\sffamily},
    >=LaTeX
    ]

\node[ct, label={[mylabel]Cell}] (ct) {$c_t$};
\node[filter, right=of ct] (int1) {};
\node[prod, right=of int1] (x1) {$\times$}; 
\node[right=of x1] (ht) {$h_t$};
\node[prod, left=of ct] (x2) {$\times$}; 
\node[filter, left=of x2] (int2) {};
\node[prod, below=5mm of ct] (x3) {$\times$}; 
\node[ft, below=5mm of x3, label={[mylabel]right:Forget Gate}] (ft) {$f_t$};
\node[ft, above=of x2, label={[mylabel]left:Input Gate}] (it) {$i_t$};
\node[ft, above=of x1, label={[mylabel]left:Output Gate}] (ot) {$o_t$};

\foreach \i/\j in {int2/x2, x2/ct, ct/int1, int1/x1,
            x1/ht, it/x2, ct/it, ct/ot, ot/x1, ft/x3}
    \draw[->] (\i)--(\j);

\draw[->] (ct) to[bend right=45] (ft);

\draw[->] (ct) to[bend right=30] (x3);
\draw[->] (x3) to[bend right=30] (ct);

\node[fit=(int2) (it) (ot) (ft), draw, inner sep=0pt] (fit) {};

\draw[<-] (fit.west|-int2) coordinate (aux)--++(180:7mm) node[left]{$x_t$};
\draw[<-] ([yshift=1mm]aux)--++(135:7mm);
\draw[<-] ([yshift=-1mm]aux)--++(-135:7mm);

\draw[<-] (fit.north-|it) coordinate (aux)--++(90:7mm) node[above]{$x_t$};
\draw[<-] ([xshift=1mm]aux)--++(45:7mm);
\draw[<-] ([xshift=-1mm]aux)--++(135:7mm);

\draw[<-] (fit.north-|ot) coordinate (aux)--++(90:7mm) node[above]{$x_t$};
\draw[<-] ([xshift=1mm]aux)--++(45:7mm);
\draw[<-] ([xshift=-1mm]aux)--++(135:7mm);

\draw[<-] (fit.south-|ft) coordinate (aux)--++(-90:7mm) node[below]{$x_t$};
\draw[<-] ([xshift=1mm]aux)--++(-45:7mm);
\draw[<-] ([xshift=-1mm]aux)--++(-135:7mm);
\end{tikzpicture}


%%% Local Variables:
%%% mode: latex
%%% TeX-master: '../quali'
%%% End:

\end{center}

As equações que regem o funcionamento da LSTM na iteração temporal $t$ são: \\


\[f_t = \sigma_g(W_fx_t + U_fh_{t-1} + b_f)\]
\[i_t = \sigma_g(W_ix_t + U_ih_{t-1} + b_i)\]
\[o_t = \sigma_g(W_ox_t + U_oh_{t-1} + b_o)\]
\[c_t = f_t \circ + i_t \circ \sigma_c(W_cx_t + U_ch_{t-1} + b_c)\]
\[h_t = o_t \circ \sigma_h(c_t)\]




%%% Local Variables:
%%% mode: latex
%%% TeX-master: "../quali"
%%% bibtex-file-path: "../bibliografia"
%%% End: